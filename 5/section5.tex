\chapter{Decision}

After surveying and trying databases in a small case, next logical step is to decide on one or poly-glot database system. We decided on running \textit{ZODB} with \textit{PostgreSQL} for main data storage. Since \textit{Redis} is very flexible and blazingly fast, it will be leveraged to offload some traffic from \textit{ZODB} and \textit{PostgreSQL}. This is the first phase of database change.

Secondly, there is one more important goal to make Indico more user-friendly and it includes:
\begin{itemize}
  \item Making Indico installable with one command
  \begin{itemize}
    \item Installing Indico for cloud now isn't easy since there is no pre-built package.
  \end{itemize}
  \item Providing Indico as SaaS \footnote{Software as a Service}
  \begin{itemize}
    \item Each installation of Indico is running independently so each one is requiring its own resources and administration. Thus, not all of servers use the latest version due to update/migration costs which gives birth to the need of support for old versions.
    \item Everybody doesn't need full featured installation of Indico for one time event, for instance.
    \item Collecting usage statistics and patterns is difficult.
  \end{itemize}
  \item To be able to use more generic accounts such as Google, Yahoo, etc.
  \item Making Indico more user-centric and social.
\end{itemize}

We have decided to use a document-oriented database for statistics collection and caching of event meta-data that will aggregated from servers around the world. Inside of available document-oriented databases, we decided on \textit{MongoDB} since it's most accessible in terms of know-how, popularity and features. Indico has a mobile version which is read-only. Thus, it's reasonable to see it as a cache for \textit{ZODB}. Mobile version uses \textit{MongoDB} since there is no need for ACID guarantee, ad-hoc queries are supported and lack of schema enables faster development. Due to nice fit between use-case and provided features by MongoDB, Indico mobile has been very successful. Data manipulation flow of new planned use-cases are very similar to mobile version. That's why \textit{MongoDB} is the selected option.

Socialization of Indico requires graph processing in which graph databases excel but at the time of writing, there is no final decision. Current main focus is to change the de-facto back-end and it will take a great amount of effort. Until it's complete, graph databases will probably change a lot and no successful big deployments with graph databases except Neo4j and custom/proprietary solutions such as Facebook. Waiting until we have to give a decision seemed to be better in that situation.

\section{Chosen Database Type}

We have first checked object-oriented databases because one good object-oriented database could reduce transition costs a lot by keeping same paradigm, enabling easy development and tight integration with Python. However, there is no good option in the market which will be able to replace \textit{ZODB}.

Column-family databases aren't considered since they are just over complex in the scale of Indico.

Key-value stores are overly simplistic. It's very difficult to convert Indico schema to key-value storage and keeping it manageable. References between entities or higher level collections are needed to easily store/retrieve relatedness of entities. However, they have specific cases where managing data is easy such as url and session caching. Since the less main database is used, the better it is; Key-value store, \textit{Redis}, is used and it'll be adapted wherever it makes sense but not for being main database.

Graph databases meet the majority of needs of Indico such as ACID transactions, replication and mapping complex relationships. However, they are niche products lacking community and every entity of complex Indico schema isn't mappable because entities are usually like trees and putting these \textit{contains} relationship into graph database exponentially increases number of edges. Therefore, storing \textit{contains} relationship into document on vertexes and using edges for far separated entities is favourable. This design is extensively supported by \textit{OrientDB}. That's why \textit{OrientDB} is studied in details but as a result, we find it to be in a fast development phase and not production ready. 

After exhausting other types, we had two options; namely, document-oriented and relational databases. This is basically a trade-off between ease-of-use (lack of strict schema and faster development) and ACID transactions. For specific cases, ACID transactions may be unnecessary because document-oriented databases are capable of supporting ACID guarantees in document level and what is multiple tables in fully normalized relational databases will be nested documents within one huge document in document-oriented databases such as room booking schema, which will be explained later as prototyping. However, when other parts started to huge documents, we can't work on document level any more, there is a need of cross references between tables.

As a result, we are left with relational databases and relational databases are the best fit for Indico use-case and schema because

\begin{itemize}
  \item Schema is composed of many cross-referenced entities which requires ACID
  \item The size of data is very manageable by relational databases
  \item Indico provides many flexible search interfaces which execute ad-hoc queries 
\end{itemize}

\section{Chosen Database}

Since we are looking for a relational database, we would take the lead of the \textit{MySQL} cloud or \textit{PostgreSQL}. The future of \textit{MySQL} cloud is confusing while \textit{PostgreSQL} is gaining popularity steadily and a much more tightly-connected community is coming life around it. In addition to these accessibility issues, getting \textit{PostgreSQL} boxes at CERN is easier via a service provided by database team at CERN \footnote{http://www.cern.ch}. Choosing \textit{MySQL} means diverging from other services at CERN which puts maintenance burden on the Indico team. We're already paying technical debt of choosing \textit{ZODB}. Moreover, \textit{ZODB} even needs little administration by being a Python library, except automated scripts such as daily packer, compared to \textit{MySQL} which is outside of Python world. Finally, \textit{PostgreSQL} has powerful features:

\begin{itemize}
  \item Most compliant DBMS with standards
  \item Most extensible and enormous number of features
  \begin{itemize}
    \item Concurrency \footnote{ACID, 64 cores}
	\item Standard SQL \footnote{fully 92-compliant}
	\item Index \footnote{bitmap index can utilize multiple indexes at the same time}
	\begin{itemize}
	  \item B-Tree
	  \item Hash
	  \item Functional
	  \item Inverted \footnote{content $\rightarrow$ row}
	\end{itemize}
	\item Triggers
	\item Schema
	\begin{itemize}
	  \item Powerful types
	  \begin{itemize}
	    \item array
	    \item IP
	    \item XML
	    \item JSON
	    \item range
	  \end{itemize}
	\end{itemize}
	\item Table Inheritance
	\item Auto Master-Slave Replication
	\item Views
	\begin{itemize}
	  \item Materialized
	  \item Modifiable
	  \item Recursive
	\end{itemize}
	\item Notifications
	\item Fine-grained security
	\begin{itemize}
	  \item Kerberos
	  \item LDAP
	  \item RADIUS
	\end{itemize}
	\item Regular Expressions
	\item Online Backups
	\item Full-text Search
	\item Administrative tasks and Analytics
	\begin{itemize}
	  \item pgAdmin
	\end{itemize}
    \item Custom background workers
	\begin{itemize}
	  \item default
	  \item pgQ
	\end{itemize}
	\item Rich plug-in ecosystem
  \end{itemize}
  \item Well-know successful deployments
  \begin{itemize}
    \item Amazon
    \item Apple
    \item Disqus
    \item Facebook via Instagram
    \item Heroku
    \item Microsoft via Skype
    \item Nasa
    \item Yahoo
  \end{itemize}
\end{itemize}

In the light of that study, \textit{PostgreSQL} is chosen to replace \textit{ZODB}.

One final remark is that \textit{PostgreSQL} was an ugly academic project in early 2000s but today, \textit{PostgreSQL} is undergoing a renaissance in features have attracted many small businesses and enterprises. Thus, knowing how to use \textit{PostgreSQL} has become an important skill set and a valuable career path. Having a open source \textit{PostgreSQL} product may have propelled the contributions.