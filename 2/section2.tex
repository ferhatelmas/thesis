% this file is called up by thesis.tex
% content in this file will be fed into the main document

\chapter{Background and Related Work} % top level followed by section, subsection

\par In this chapter, we first look at some topological aspects of today's OSN, in the context of their impact on the structure of the underlying P2P infrastructure. We then focus on specific ways of offering such infrastructure, and focus on the SpiderCast protocol. We then survey the related work on building P2POSN, discuss their approaches, and point out their connection to our own work, when applicable. Finally, we present some insights about the global characteristics of the social graphs and P2P network graphs, which further motivate our work.  
          

\section{Online Social Networks}

\par The inherent characteristics of OSNs, as they were implemented in the currently available client-server architectures, should be considered when designing a P2P platform able to sustain an OSN with a large userbase. They have been comprehensively studied following (a) a purely \textit{topological view} \cite{Mislove, kumar}, focusing, for instance, on the structural position of their nodes, on their level of segmentation or connectivity, on the structural mechanisms that describe their evolution, etc; and (b) a \textit{semantic view} \cite{twitter, java}, which brings together structural characteristics and user data patterns (e.g. what type of information has a tendency to be further shared/retweeted and by what category of users, which are the most popular topics or users). All these characteristics may be used not only to help design the underlying architecture of OSN, but also to enhance P2P systems, recommendation systems, location based systems, trust computation and so on.

\par In the context of this work, we will look at the former category of studies, because we are mainly interested to see how the topology of the P2P platform impacts the performance at the OSN level (e.g. a user may not want to keep links to any other users and/or to have no links to his friends). For example, the first OSN is considered to be created by the users who exchanged email messages, as a social graph was formed in this way. Today, popular OSNs, such as Facebook \cite{web1}, Twitter \cite{web2}, LinkedIn \cite{web3} and Flicker \cite{web4}, rely on such graphs to perform basic operations like sharing, organization or data location \cite{Mislove}. 

\par It was observed that, for any OSN, the social graph is formed from three different types of groups (i.e. sets of nodes) \cite{kumar}: (a) \textit{zero-degree nodes} - usually not having an active participation within the network, (b) a \textit{giant connected component} - densely connected social network core, containing a large number of users and having at least one path between any pair of nodes (typically containing high-degree and highly active nodes), and (c) \textit{isolated communities} - small sets of nodes who interact only with each other. Note that in time the giant connected component tends to gather the vast majority of users \cite{Mislove,mascolo}, up to 99\% of nodes (as the isolated communities merge with it \cite{kumar}). This property holds for all the networks used for our system evaluation, being also preserved with high probability by the obtained overlay, as we will see in Chapter~4. We will further analyze all these characteristics and more, making also a short comparison with P2P networks characteristics, in Section 2.4.


%There is an extremely large number of publications that analysis today's popular online social networks phenomenon and their characteristics such those previously enumerated \cite{Mislove, facebook, social, twitter}. Throughout this work we will consider many of these studies when making design decisions and designing relevant and realistic test cases. Thus, in the following fifth section of this chapter we will explain in more details those relevant for our work. 



\section{P2P Networks} 

\par This thesis argues about the feasibility of a P2P system that aims to minimize as much as possible the users' tradeoffs when moving away from a social network service based on the client/server paradigm. Thus, when architecting a P2P-based OSN, we also need to deal with P2P-specific aspects, such as data availability, updates, network topology, search, data locality, etc. For this work, we mainly focus on obtaining data availability, scalability, and inexpensive communication. In this section we will point out only aspects related with this goal.

%The requirements for architecting a P2P-based OSN are related both to P2P-specific issues, as well as OSN characteristics. On one side, we need to take care of data storage and updates, network topology, search and data locality, while on the other we should face the important volume of social data typical for any popular OSN. Furthermore, when having nodes with hundreds or maybe thousands of friends in the social network, it is very costly to maintain connections with each of them in order to propagate updates (wall posts, pictures uploading, etc.).

\par The P2P model was made popular mostly by file-sharing and content distribution applications such as BitTorrent\cite{p2p1}, eDonkey\cite{p2p4}, Napster\cite{p2p2}, Gnutella\cite{p2p3}, etc. Apart from this, it is also used for real time communication between users. The nodes in a P2P network form overlays which can be classified as structured and unstructured. Even if unstructured overlays are often relatively simple, their search operations tend to be inefficient. On the other hand, structured P2P overlays usually use distributed hash tables (DHT) to perform directed searches, which makes lookups more efficient in locating data \cite{search2006, search2007}. Thus, for search performance, a structured overlay may be preferred. However, they tend to be more expensive to maintain especially under high churn rate, as their topology impose more constraints (which makes one think of the potentially large control message overhead and degraded routing performance). Unstructured overlays are arguably more resistant to churn, but it is commonly believed that they impose greater message overhead (as they generally use either flooding or random walks)\cite{Qiao,Castro}. On the other hand, when the overlay network exhibits the clustering and short path lengths of a small world network, it ensures that even message flooding approaches work well (with a low time-to-live for messages, resulting in a low message overhead) \cite{Aberer,Stutzbach}. 

\par Thus, this motivates the following question: how to propagate updates while maintaining an overlay which is scalable, churn-resistant, small diameter and with a low maintenance cost for connections. Basically, as we presume that nodes in a P2P social network will primary store the data they publish on, we need a solution for building an overlay network that allows efficient topic-based event routing in a highly dynamic environment. In this regard, the SpiderCast protocol supports publish/subscribe communication in a decentralized environment \cite{spidercast}. This protocol aims to build an overlay that is both topic-connected and has a low per-topic diameter, while requiring each node to maintain a low average number of connections. In our P2P social network case, the topics are the same with the nodes' profiles, hence the number of topics is equal with the number of nodes and each node interest contains only his friends. Thus, this protocol fits our needs since we can see the P2POSN as a publish/subscribe system in which each user subscribes to his friends. In addition, we can efficiently use the principles behind SpiderCast protocol to reduce the average node degree by manipulating the common interests (friends) between nodes for constructing a low-degree overlay in which all the nodes with a common friend form a connected component. Since the clustering coefficient is known to be high in social networks \cite{Mislove,facebook} we expect to have a large number of duplicate paths, a good premise for SpiderCast to be successful in this environment. Furthermore, the experiments on SpiderCast show that the average node degree decreases with the number of nodes \cite{spidercast}. When keeping the number of nodes constant, the increase of the number of topics moderately increases the average node degree, demonstrating also a good scalability with the number of topics.

\par In our context, we should keep in mind that the availability of any specific file is not guaranteed in a P2P network. We will see that this can become a problem for a P2P social network, as for today's OSNs this is a basic requirement. Thus, we need a cost efficient solution able to ensure data accessibility with a high probability. Towards obtaining high availability for distributed application in an environment prone to a high churn rate, a solution for self-healing active replication on top of a structured P2P overlay is proposed in \cite{semias}. Yet, each data item is associated with a unique key and its replicas are placed on the closest nodes to its key. This may not be the best approach for us as we might need to consider, for replication, node attributes such as: users diurnal behavior, number of friends, access patterns, interaction patterns, or we should provide a mean of computing the key so as to consider these attributes when computing the distance between keys, and is challenging to consider all of them at once.


\section{P2P Social Networks}

\par A P2P network can be seen as a social network where users are the nodes linked according to their friendship relations; the similarities between the two have been previously explored in \cite{case,futureosnp2p}. Such networks become as large as hundreds of millions of nodes. The authors try to identify a pertinent set of opportunities and challenges for a P2P environment in order to support the deployment of a social network. Since this identification was made, services like Diaspora \cite{webp2pd1}, considered a Facebook's young competitor and Identi.ca \cite{webp2pd2}, a solution to decentralize Twitter, have started to appear.

\par Solutions for P2P social networking have come also from the research community. Gossple, which links users based on their similarities, is a network of anonymous social acquaintances that uses a gossip protocol for updates and communication \cite{gossple}. Gossple provides a way to meaningfully associate and exchange information between users' profiles, without requiring peers to have a previous relation. This is an interesting approach as one of the reasons for belonging to a particular social network may be to meet people. Thus, a user may seek to meet people with similar interests, passions, background, problems, professions and hobbies, shifting the key focus on finding like-minded people. Basically, the nodes periodically gossip data about their interest profiles and based on this interest compute the distances between them and other nodes and create links. However, our work differs in a number of aspects as we do not look into preserving users' anonymity and to creating links based on a defined interest metric. 

\par An approach closely related with ours is PeerSoN, which aims to overcome the OSNs limitations by looking to find solutions to the privacy issues (i.e. encryption, access control and decentralization) and permitting direct exchange of data between users even without Internet connectivity, while preserving OSNs features (e.g. data availability and accessibility) \cite{peerson}. For ensuring that the private information is visible only by the right users, PeerSoN does not assume any kind of trust relationship between peers, but provides access control by using encryption and key management. While similar solutions could be adapted to the work presented in this thesis, we believe that we first need to define the design of a scalable P2POSN underlying infrastructure. 

\par LifeSocial.KOM is a P2P-based platform for secure online social networks which claims to provide the functionality of common online social networks in a totally distributed and secure manner \cite{kom}. It uses FreePastry for interconnecting the participating nodes and PAST for reliable, replicated data storage. Alas, it was tested at a very small-scale only. 

\par In \cite{porkut} a privacy aware decentralized OSN called Porkut is introduced. The idea behind it is to exploit trust relationships in the social network for decentralized storage of OSN profiles and to address availability and accessibility issues by considering users' geographical locations and online time statistics, while using k-anonymity techniques for ensuring a privacy preserving indexing. Yet, they store the data in plain text relying on social pressure and monitoring for data integrity and privacy. A solution, for providing reliable storage, is to offer proper incentives to nodes, based on existing social relationships, in order to make them to cooperate for storing data and to remain in the system \cite{f2f}. However, this approach is suitable mainly for data intensive applications that do not require global visibility or global sharing. 

\par There are also customized solutions such as SCOPE which is a prototype for spontaneous P2P social networking \cite{scope}. It provides the distributed database and lookup services deployed on DHT technology so as to support social networking in local areas. Finally, the Maze P2P network integrates both a friends' social network and a download network. The similarity between nodes' interest is revealed by the social network relationship within the Maze network \cite{maze}. Thus, Maze is improves the content search in the download network by using the ``small world'' property \cite{small} of the friends' social network as it enables small search path length.

 
\section{Brief Analysis of Social and P2P Networks Graphs}

\par When trying to determine if a shift from a client/server to a P2P infrastructure is feasible, all the characteristics of today's OSNs such as clustering coefficient, path length, access patterns, interaction patterns, etc., need to be considered so as to make a realistic feasibility study.

 \begin{table}[t]
    \begin{tabular}{ | p{4.6cm} | p{4.0cm} | p{4.0cm} |}
    \hline
    \textbf{Network} & \textbf{Average path length} & \textbf{Type (OSN or P2P)}\\ \hline
    Facebook\cite{facebook} & 4.8 & OSN  \\ \hline
    Flicker\cite{Mislove} & 5.67 & OSN  \\ \hline
    Twitter\cite{twitter} & 4.12 & OSN \\ \hline
    Maze friends network\cite{maze} & 7.69  & P2P  \\ \hline
    Maze download network\cite{maze} & 9.17 & P2P \\ \hline
    \end{tabular}
    \caption{Average path lengths for OSNs and P2P networks}
    \label{small_tab}
\end{table}

\par Social graphs exhibits the ``small world'' property which practically measures to what extent a network can spread information quickly and widely \cite{small}. It is obtained with a small average path length and a high clustering coefficient. We observe that while ONSs have their average path length lower than that computed for world population [6.6] \cite{six}, the average path length for P2P networks is higher. The average path length values for some of the most popular social networks and some P2P networks are presented in Table \ref{small_tab}. The values for average path lengths and clustering coefficients obtained for all these graphs (so as for any ``small world'' network) compared with those obtained for a corresponding random network, with the same number of nodes and average degree per node, have an average path length close to the random network, while the clustering coefficient is much larger \cite{small}. The data in these papers suggests that even if for P2P networks the path length is slightly bigger and the clustering coefficient slightly lower, they seem to have similar topological structure. More, with redundancy and some solution for information availability P2P social networks can reach ``six degrees of separation" and in a P2P network we can take advantage of this short average path length. For example, flooding with relatively small time to live would cover most of the graph. In addition, if this is correlated with a large clustering coefficient, interest based clusters of peer can be identified. These features can influence the performance for operations such as information dissemination, friends search, etc., in a P2P social network.
 
 
 
 %\par The first observation is that when the average number of friends per user is increasing the average path length has a bigger probability to decrease, because the probability of intersection between groups of friends is increasing. Second, when there are asymmetric relations as in Twitter, where users follow other users to obtain information, the path length will also decrease \cite{twitter}. 

%\par In fact the small values of average path length of ONSs are almost certainly obtained because of the unnatural characteristics of these networks in comparison with real life relation. For the same reason a large clustering coefficient is improbably to be obtained with real life social relations. In addition, in the case of Maze \cite{maze}, which use peers social relations, can be observed that the obtained numbers for path length are bigger. 

\par Some very interesting findings show that in unstructured Gnutella-like P2P overlays: a) the long-lived peers form a densely connected and stable core, which leads to stable and high connectivity among peers even in the presence of high churn and b) the network dynamics lead to a core overlay that exhibits �onion-like� biased connectivity (peers form ``layers'' depending on their uptime - peers with shorter uptime form a layer by forming connections with each other and with peers with higher uptime) \cite{Stutzbach}. This is quite similar with how the OSNs evolves in time: old users have a tendency of having larger friendbase and to move closer to the social network core and to also form a densely connected core \cite{kumar}.

\par  Considering only the ``static'' social graph does not suffice. A proof of the fact that social links are not valid indicators of real users' interactions, for the biggest OSN Facebook, is given in \cite{facebook}. There is a high skew distribution of users interaction, for instance less than 1\% of Facebook users interact with more than 50\% of their friends. Consequently, the social links should include data about users' interactions. For 50\% of users, 70\% of interaction comes from 7\% of friends and 100\% of interaction comes from 20\% of users. Basically, this also mean that the interaction graph can eliminate useless links in the ``static'' social graph, as the results correspond with theoretical limits on human social cognition\footnote[1]{ This limit is given by Dunbar's number (150) which represents the maximum number of stable social relations that one can maintain.} \cite{small}. These invalid links are the reason why, for instance, the OSNs exhibit lower values for path length than those obtain for world population. 99\% of Facebook users interact only with less than 150 friends. While the interaction links weight (i.e. the number of interactions between to users) could be used to further enhance our P2POSN system design by giving priority to stronger links (i.e. links with higher weight), this work focus on keeping all node's friends profiles accessible, regardless the interaction patterns. %In Chapter 4 we will discus and compare results obtained on both ``static'' and interaction social graphs.

\par However, while users' visible interaction links provide clues on how to design a social platform, the majority of user interactions in OSNs are latent interactions (e.g. browsing a user profile) \cite{latent}. This study is made on Renren, the largest OSN in China, on 42 million users, 1.66 billion social links and detailed histories of profile visits over a period of 90 days for more than 61,000 users. It analysis the latent interaction graph which basically models users' browsing behavior. In addition, it shows that the latent interaction graph has a clustering coefficient (0.03) lower than those obtained for social graph (0.18) and visible interaction graph (0.05), while in terms of connectivity it falls between them \cite{latent}. These characteristics could be relevant for us mainly for data location and replication. They allow the weighting of all the social links, in order to give priority to some particular links as well. Yet, this extension is beyond the scope of our work.

 






% ---------------------------------------------------------------------------
% ----------------------- end of thesis sub-document ------------------------
% ---------------------------------------------------------------------------