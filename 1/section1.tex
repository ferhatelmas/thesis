\chapter{Introduction}

\section{Overview - What is Indico?}

Indico (Integrated Digital Conferencing) is a web application for event management, initially developed as a joint initiative of \textsc{CERN}\footnote{\href{http://home.web.cern.ch/}{The European Organization for Nuclear Research}}, \textsc{SISSA}\footnote{\href{https://www.sissa.it/}{International School for Advanced Studies}}, University of Udine\footnote{http://www.uniud.it/}, \textsc{TNO}\footnote{\href{https://www.tno.nl/}{Nederlandse Organisatie voor Toegepast Natuurwetenschappelijk Onderzoek}}, and University of Amsterdam\footnote{\url{http://www.uva.nl/en/home}}. Currently, most development is happening at CERN with occasional contributions from the outside, such as time zone support by Fermilab\footnote{\url{https://www.fnal.gov/}}.

\par Indico events are divided into 3 types: lecture, meeting and conference as from simplest to the most complex, respectively. Lectures are one time talks while conferences provide a wider range of features.

\par In addition to events having a type, each event belongs to a specific category, which enables category-driven navigation between events. However, active categories can contain many events in which finding events may be cusbersome so time-driven (day/week/month) view is combined with category view for a more powerful navigation. Sometimes time-driven navigation isn't enough, as any user, extensively uses Indico, can easily end up with tens of events in a daily view. To remedy this problem, personal dashboard was developed in which upcoming events are shown in sorted order to the user. Moreover, users can mark some categories as their favorites and Indico recommends categories to users by analyzing this information and the event history of users.

\par Indico has a huge number of features and some important ones:
\begin{itemize}
  \item Customization of the event's website.
  \item Easily setting up access and modification rights: Every part of conference management requires different access rights. One contributor should be able to modify her contribution but she shouldn't be able to change others, for instance.
  \item A powerful fully customizable WYSIWYG registration form for events
  \item Custom forms such as preparing a survey to get feedback about event.
  \item Call for abstracts
  \item Paper reviewing
  \item Rich text editing in Latex and Markdown
  \item Archiving files: pictures, slides, videos, etc. It's a document store as much as it's a meta-data store.
  \item Converting documents between different formats via an external conversion server such as automatically generating PDF from Microsoft Office slides.
  \item Exporting event meta-data in multiple formats: HTML, XML, iCal, PDF, etc.
  \item Extensible plug-in system to provide more advanced features and to integrate with other systems. Some widely-used plug-ins:
  \begin{itemize}
    \item A powerful Room Booking system which enables management and reservation of rooms in a large organization
    \item Powerful video conferencing by integration with Vidyo\footnote{\url{http://www.vidyo.com/}} and EVO \footnote{\url{http://evo.caltech.edu/}}
    \item Using instant messaging, XMPP, is easy from Indico
    \item Payment processing for registration using different providers (PayPal, PostFinance, WorldPay, etc.)
    \item Search for categories and events via Invenio\footnote{\href{https://invenio-software.org/}{Invenio}, also developed at \textsc{CERN}, is a digital management library by providing classification, indexing and curation of documents} integration
  \end{itemize}
  \item Fully embracing mobile with mobile version and mobile check-in application for events
\end{itemize}

As seen, Indico is absorbing new feature sets by integrating with other services and steadily advancing to become the de facto interface for most of the needs in terms of event organization and management.

\section{Motivation - Why change?}

\par The main driving force in development of Indico was the need for a service to manage events, a successor of CERN's own CDS\footnote{\href{http://cds.cern.ch/?ln=en}{CERN Document Server}}. Thus, rapid development of production ready software was a necessity. Since there were many features to be implemented, time spent in other parts of the system had to be minimized.

\par The beginning of the 2000s was the time objects and object oriented programming languages became mainstream. As a result, there was a tendency to try objects everywhere and data stores had their own lion's share. Thus, an object database as a back-end for web application was a reasonable decision at that time. However, object databases never took off as general-purpose solutions.

\par \textsc{ZODB} (Zope Object Database) was a viable option for Python in 2002. Since \textsc{ZODB} is an object database, it enables direct manipulation and transparently storage of objects. This capability of certain data stores eliminates a layer between objects in the programming language and data in store which in turn saves a lot of time, money and complexity in development.

\par Fast forward to 2014, \textsc{ZODB} is far from having a medium sized community. What was saved before by using \textsc{ZODB} is now being paid to work around its own limitations. In the last couple of years, the net gain turned out to be negative which meant replacing \textsc{ZODB} with a more standard, powerful and scalable data store which will be the basis for new developments in the next decade of life of the tool.

\section{Approach - How to change?}

\par We are sure about the necessity of the change but change is costly and getting costlier every day because more feature are being developed, more systems are being integrated and scale is increasing. Thus, a wrong decision taken today may even kill the development of the tool so it should be well-planned: past mistakes should be analysed to prevent them happening again and emerging technologies should be examined to utilize.

\par Firstly, limitations of \textsc{ZODB} should be extensively studied because most of the time, leaving aging production system behind and writing it from scratch results in failure \cite{rewrite-cost}\cite{rewrite-cost-2}. It's developed for more than 10 years, there is a huge body of accumulated expertise with working \textsc{ZODB}. As a result, many work around solutions are developed and which should be studied to make clear where it failed. Unless problems are made ridiculously clear, clean solutions can't be found\cite{pracprog}.

\par We will present these reasons and how we tried to solve these limitations as in the existing system and technologies. After becoming certain that refactoring would be more cost-effective instead of pushing \textsc{ZODB} more and more, next step is to define features, that new data layer should be capable of. We have mainly used two things: where \textsc{ZODB} failed and interpolation of the growth.

\par In the golden era of data stores, started by Google\cite{bigtable}\cite{chubby}\cite{mapreduce} and Amazon\cite{dynamo} who needed to implement their own systems with extreme requirements; there are zillions of candidates, and even barely knowing each of them is a really difficult task. The goal is to collect and classify candidates that conform to criteria as much as possible.

\par After a first elimination and classification, the next step is to get familiar with each one via implementing a very small subset of Indico. Implementation is important because it is hard to estimate from abstract concepts alone. Even if the feature to be implemented is small, it makes it easier to predict the schedule and complexity.

\par When know-how about each system is acquired, the tipping point is choosing one or a subset of the candidates because, in big applications, each part has different characteristics and requirements which entail suboptimal solutions when using only one data store.

\par In either case, the transition phase is long and painful so it should be well-planned in terms of bugs should be fixed immediately, needs of the community and release cycle. Indico is an open source project and there is a vibrant community engaging worldwide. Thus, unlike a closed-source product, decisions can't be given only by \textsc{CERN}, the final decision maker, though. Once the schedule is carefully studied, the final bit is to implement the system conforming to the plan.

\section{Outline - What is Done and Next?}

\par In the following chapter, we will provide an exhaustive description of the work done from the study of limitations of ZODB to a new implementation using a new polyglot database system. Moreover, database change involves refactoring a huge body of code base so this project also aims to pay years of technical debt and to complete what we wish we had had in the beginning, such as a fully modular Indico. In conclusion section, how important steps are taken to modernise Indico code base and to prepare transition of Indico to Python 3 can be seen in addition to getting scalable by new back-end.

\par Even if important steps are accomplished, that's not enough because software is very volatile and requirements and technologies are quickly changing. While keeping basic principles like DRY and KISS, extending data system for specific cases will help with better scalability and facilitate the development of certain functionality such as statistics and social features.

