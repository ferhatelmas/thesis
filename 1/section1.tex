
% this file is called up by thesis.tex
% content in this file will be fed into the main document

%: ----------------------- introduction file header -----------------------
\chapter{Introduction}

% the code below specifies where the figures are stored
\ifpdf
    \graphicspath{{1_introduction/figures/PNG/}{1_introduction/figures/PDF/}{1_introduction/figures/}}
\else
    \graphicspath{{1_introduction/figures/EPS/}{1_introduction/figures/}}
\fi

\section{Overview - What is Indico?} % section headings are printed smaller than chapter names

Indico (Integrated Digital Conferencing) is a web application for event management, initially developed by a joint initiative of CERN, SISSA, University of Udine, Nederlandse Organisatie voor Toegepast Natuurwetenschappelijk Onderzoek (TNO), and University of Amsterdam. Currently, main development is going on at CERN with occasional contributions from outside such as time zone support by Fermi Lab.

\par Indico events are divided into 3 types: lecture, meeting and conference as from simplest to the most complex, respectively. Lectures are one time talks but conferences provide all nice features of Indico.

\par In addition to events having a type, each event belongs to one category which enable category-driven navigation between events. Since finding interested event in possibly huge number of categories, time-driven (day/week/month) view is also supported. Even time-driven navigation isn't enough because a user, extensively uses Indico, can easily see tens of events in a daily view. To remedy this problem, personal dashboard is being developed in which approaching events are shown in sorted order for the user. Moreover, users can mark some categories as their favorites and Indico recommends categories to users by analyzing this information and the event history of users.

\par Indico manages full conference cycle starting with customization of website. Every part of conference management requires different access rights. One contributor should be able to modify her contribution but she shouldn't be able to others, for instance. Indico makes setup of access and modification rights easy. Conferences are nothing without registrants and Indico provides a powerful fully customizable wysiwyg registration form. Indico supports creating custom forms in addition to registration form such as to prepare a survey. Call for abstracts, rich text editing in Latex and Markdown, and paper reviewing are enabled by Indico. Indico is a document store as much as being metadata store but saving different kinds of files; pictures, slides, videos, etc. Indico is even capable of converting documents betweeen different formats such as automatic PDF generation from Microsoft Office slides. Everybody has her different flow and tools to manage her calendar. Indico allows different views of same event and exporting events in multiple formats such as HTML, XML, iCal, PDF, etc. Last but not least, Indico is extensible via plug-ins to integrate with other systems or to provide more localised services so that only Indico is sufficient for everything. Room booking to arrange event place or Vidyo for video conference may be given as examples of notable plug-ins.

\par Extensible architecture of Indico have enabled easy integration with different services. As time advanced, search engine (Invenio) is added to find catgories and events easily. Indico totally embraces mobile and recently mobile check-in application is developed. Video conference (Vidyo and EVO) and instant messaging (XMPP) are easy to use within Indico. Payment processing for registrations in different ways (PayPal, PostFinance, WorldPay, etc.) is possible. Location, room and booking management are made possible with a clear interface for places and schedule of events. As a result, Indico is absorbing new feature sets by integrating with other services and steadily advancing to become the de facto interface for most of the needs.

\section{Motivation - Why change?}

\par Main driving force in development of Indico was a required service to manage events as following agenda of CDS, CERN Document Server. Thus, rapid development of production ready software was a necessity. Since there were many features to be implemented, time spent in other parts of the system must have been minimized.

\par The beginning of 2000s was the time objects and object oriented programming languages have become mainstream. As a result, there was an opportunist storm to try objects in everywhere and data stores had their own lion's share. Thus, an object database as a back-end for web application was a meaningful decision at that time. However, object databases never took off.

\par ZODB (Zope Object Database) was a viewable option in 2002 for Python, main programming language of Indico. Since ZODB is an object database, it enables directly manipulating and transparently storing objects. This capability of data store eliminates a layer between objects in programming language and data in store which in turn, saves a lot of time, money and complexity in application.

\par Fast forward to 2014, ZODB is far from having medium sized community. What is saved before by using ZODB is now being paid to solve limitations of ZODB. Finally, net gain turned into negative which signified to replace ZODB with more mainstream, powerful and scalable data store which will support next decade of Indico.

\section{Approach - How change?}

\par Firstly, limitations of ZODB should be extensively studied because most of the time, leaving aging production system behind and writing it from scratch results in failure. It's developed for more than 10 years, leaving ZODB means discarding this accumulated knowledge. Moreover, unless problems are made ridiculously clear, clean solutions can't be found.

\par We will present these reasons and the ways how we tried to solve these limitations in existing system. After we become certain about that refactoring would be more cost-effective, next step is to define criterias for the replacement technology in terms of where ZODB has failed and also keeping the future of Indico in mind. 

\par In the golden era of data store, started by Google and Amazon to implement their extreme requirements, there are zillions of candidates, even barely knowing each of them is a really difficult task. Goal is to collect and classify candidates that conform to criterias as much as possible.

\par After first elimination and classification, next round is to get familiar with each one via implementing a very small subset of Indico. Implementation is important because making sense of abstract concepts is difficult to interpolate expectations. Even if feature implemented is small, it makes easier to predict the schedule and complexity.

\par When know-how of each system is acquired, tipping point is choosing one or subset of candidates because,  in big applications, each part has different characteristics and requirements which maps into suboptimal solutions by using only one data store.

\par In either case, transition phase is long and painful so it should be well-planned in terms of bugs, needs of community and release cycle. When schedule is carefully studied, the final bit is to implement the system conforming to the plan.

\section{Outline - What is Done and Next?}

\par In the following, we will give every tiny detail from the study of limitations of ZODB to new implementation with new polyglot database system. Moreover, database change involves refactoring huge body of code base so this project also aims to pay years of technical debt and to complete what we wish we had had in the beginning such as modular Indico. In conclusion section, how important steps are taken to modernise Indico code base and to prepare transition of Indico to Python 3 can be seen in addition to getting scalable by new back-end.

\par Even if important steps are accomplished, that's not enough because software is very volatile and requirements and technologies are quickly changing. While keeping basic principles like DRY and KISS, extending data system for specific cases will help for better scalability and facilitate the development of certain features such as statistics and social features.

