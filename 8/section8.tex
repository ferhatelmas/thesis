\chapter{Conclusion}

Indico is nowadays present in the lives of thousands of CERN employees and users. Since the beginning, usage has grown steadily, which brought in new challenges. One of these challenges is that of the database back-end. ZODB couldn't cope with such high traffic and several features and improvements required complex workarounds such as application level indexing, usage of Redis as a caching layer and separating costly query flows into an independent database as it was done in the room booking module. These improvements were satisfactory but the root problem wasn't solved, being just delayed for some time. The development of new features was getting costlier with every passing day. The solution, a database change, was obvious but also very daring. That is what kept it from being seriously considered (until now).

A database change was imperative, but the choice of the new backend had to be carefully considered, since:
\begin{itemize}
  \item A database change is very expensive
  \item There are many candidates of very different types/families
  \item There are slight differences in features among candidates
  \item A drop-in replacement is seldom possible
  \item The future is not certain regarding the evolution of Indico's feature set over the next decade
\end{itemize}

Taking these issues into account, an extensive study was carried, which included a comparison between different candidates. PostgreSQL was chosen.

It was also decided to incrementally extract modules and port them to the new database infrastructure, so as to make an aging and complex code base more modular and extensible.

At the time of writing, the room booking module is already using a different database. This module executes heavy and complex queries, which makes it a good starting point. It is also quite autonomous from the main data store, thus comparably easier to isolate.

Incremental development started in the module in question, and since ZODB and PostgreSQL will coexist until the transition is fully complete, the required distributed query machinery was put into place.

At the same time, the code structure is being changed in order to conform with the MVC pattern. The request processing work flow is delegated to libraries as much as possible, which leverage the power of Flask (web) and SQLAlchemy (back-end).

During this porting effort, PostgreSQL has already shown that it is a natural fit for Indico's schema. Expected performance and storage size improvements are already visible. Even if PostgreSQL performs quite well by itself, Redis is still being used in front of PostgreSQL, in order to reduce the load and provide memory-stored results rather than spending database time.

Very important and decisive steps have been taken so far but there is still a huge body of code waiting to be renovated.
