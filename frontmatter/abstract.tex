
% Abstract -------------------------------------------------------------


%\begin{abstractslong}    %uncommenting this line, gives a different abstract heading
\begin{abstracts}        %this creates the heading for the abstract page

\par Indico is a web application to schedule and organise events, from simple lectures to complex meetings, workshops and conferences with sessions and contributions. The tool also includes advanced user delegation mechanism, paper reviewing, archiving of event materials. More tailored features such as room booking and collaboration are provided via plug-ins.
\\
\par Indico is heavily used at CERN and in more than 10 years, very different features are added according to the needs of increasing number of users. However, data store(ZODB) have never changed. ZODB has been very important for Indico to have been Indico but it's a bottleneck now in terms of scalability and elapsed time in feature development.
\\
\par Indico demands more scalable backend to serve increasing number of events with fancy features. The aim of the project is to replace ZODB with a highly scalable data layer which will enable global Indico, bringing $\sim$120 different Indico servers around the world together. First part of project involves the survey of different technologies according to the needs of Indico and second part is composed of preparation of transition environment with a prototype.
\\
\\
\textbf{Keywords}: SQL, NoSQL, Python, Web Development

\end{abstracts}

%\end{abstractslong}


% ---------------------------------------------------------------------- 
