\begin{abstracts}

\par Indico is a web application which is used to schedule and organise events, from simple lectures to complex meetings, workshops and conferences with sessions and contributions. The tool also includes an advanced user delegation mechanism, paper reviewing, archiving of event materials. More custom tailored features such as room booking and collaboration are provided via plug-ins.
\\
\par Indico is heavily used at \textsc{CERN}\footnote{\url{http://home.web.cern.ch/}} and, in more than 10 years, very different features were added according to the needs of an even increasing number of users. However, the data store, \textsc{ZODB}, has never changed. \textsc{ZODB} has been very vital for the fast development of many features of Indico but it's now a bottleneck in terms of scalability and elapsed time in feature development.
\\
\par Indico currently requires a more scalable back-end to serve an increasing number of events while providing advanced features. The aim of the project is to replace \textsc{ZODB} with a highly scalable data layer to face this demand. The first part of project involves the analysis of different databases in the light of Indico schema, scalability and migration cost. The second part is bringing chosen database into production by development of a prototype that works in an incremental transition environment.
\\
\\
\textbf{Keywords}: SQL, NoSQL, Python, Web Development, CERN

\end{abstracts}
