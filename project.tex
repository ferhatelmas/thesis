% ----------------------------------------------------------------------
%                   LATEX TEMPLATE FOR PhD THESIS
% ----------------------------------------------------------------------

% based on Harish Bhanderi's PhD/MPhil template, then Uni Cambridge
% http://www-h.eng.cam.ac.uk/help/tpl/textprocessing/ThesisStyle/
% corrected and extended in 2007 by Jakob Suckale, then MPI-CBG PhD programme
% and made available through OpenWetWare.org - the free biology wiki
   
 
%: Style file for Latex
% Most style definitions are in the external file PhDthesisPSnPDF.
% In this template package, it can be found in ./Latex/Classes/
\documentclass[oneside,11pt]{Latex/Classes/PhDthesisPSnPDF}
  

%: Macro file for Latex
% Macros help you summarise frequently repeated Latex commands.
% Here, they are placed in an external file /Latex/Macros/MacroFile1.tex
% An macro that you may use frequently is the figuremacro (see introduction.tex)
\include{Latex/Macros/MacroFile1}

\usepackage[normal]{subfigure}
\usepackage[left=3cm, bottom=3cm]{geometry}
\usepackage{listings}
\usepackage[utf8x]{inputenc}
\usepackage{ucs}
\usepackage{bbding}
\usepackage{array}
\usepackage{booktabs}
\usepackage{float}
%: ----------------------------------------------------------------------
%:                  TITLE PAGE:
% ----------------------------------------------------------------------
% below is to generate the title page with crest and author name

%if output to PDF then put the following in PDF header
\ifpdf  
    \pdfinfo { /Title  (Indico: Behind the Scenes of CERN events)
               /Creator (TeX)
               /Producer (pdfTeX)
               /Author (Ferhat Elmas ferhat.elmas@epfl.ch)
               /CreationDate (D:20140314)  %format D:YYYYMMDDhhmmss
               /ModDate (D:20140314)
               /Subject (Master Thesis - DB Project)
               /Keywords (SQL, NoSQL, Python, JavaScript, Web Development, GNU/GPL, Scalability) }
    \pdfcatalog { /PageMode (/UseOutlines)
                  /OpenAction (fitbh)  }
\fi


\title{Indico: Behind the Scenes of CERN events}

 
% ----------------------------------------------------------------------
% The section below defines www links/email for author and institutions
% They will appear on the title page of the PDF and can be clicked
\ifpdf
  \author{\href{mailto:ferhat.elmas@epfl.ch}{\textbf{Ferhat Elmas}}\\ 
  \href{mailto:pedro.ferreira@cern.ch}{CERN Supervisor: \textbf{Pedro Macedo Alves Ferreira}}  \\
  \href{mailto:karl.aberer@epfl.ch}{EPFL Supervisor: \textbf{Karl Aberer}}  \\
  }
  \collegeordept{\href{http://ic.epfl.ch}{School of Computer and Communication Science}} 
  \university{\href{http://www.epfl.ch}{EPFL}}
  \crest{\includegraphics[width=4cm]{EPFL.png}} 
  
% If you are not creating a PDF then use the following. The default is PDF.
\else
  \author{Ferhat Elmas}
  \cityofbirth{Supervised by: Pedro Macedo Alves Ferreira \and Karl Aberer}
  \collegeordept{School of Computer and Communication Science}
  \university{EPFL}
  \crest{\includegraphics[width=4cm]{epfl.png}}
\fi   
  
\renewcommand{\submittedtext}{Master Project Report}  
\degree{Distributed Information Systems Laboratory}
\degreedate{14 Mar 2014}


% ----------------------------------------------------------------------
       
% turn of those nasty overfull and underfull hboxes
\hbadness=10000
\hfuzz=50pt


%: --------------------------------------------------------------
%:                  FRONT MATTER: dedications, abstract,..
% --------------------------------------------------------------
\bibpunct{[}{]}{,}{n}{,}{,}

\begin{document}

%\language{english}

% sets line spacing
\renewcommand\baselinestretch{1.0}
\baselineskip=14pt plus0.5pt
 
%: ----------------------- generate cover page ------------------------

\maketitle  % command to print the title page with above variables


%: ----------------------- abstract ------------------------

% Your institution may have specific regulations if you need an abstract and where it is to be placed in the document. The default here is just after title.


% Abstract -------------------------------------------------------------


%\begin{abstractslong}    %uncommenting this line, gives a different abstract heading
\begin{abstracts}        %this creates the heading for the abstract page

\par Indico is a web application to schedule and organise events, from simple lectures to complex meetings, workshops and conferences with sessions and contributions. The tool also includes advanced user delegation mechanism, paper reviewing, archiving of event materials. More tailored features such as room booking and collaboration are provided via plug-ins.
\\
\par Indico is heavily used at CERN and in more than 10 years, very different features are added according to the needs of increasing number of users. However, data store(ZODB) have never changed. ZODB has been very important for Indico to have been Indico but it's a bottleneck now in terms of scalability and elapsed time in feature development.
\\
\par Indico demands more scalable backend to serve increasing number of events with fancy features. The aim of the project is to replace ZODB with a highly scalable data layer which will enable global Indico, bringing $\sim$120 different Indico servers around the world together. First part of project involves the survey of different technologies according to the needs of Indico and second part is composed of preparation of transition environment with a prototype.
\\
\\
\textbf{Keywords}: SQL, NoSQL, Python, Web Development

\end{abstracts}

%\end{abstractslong}


% ---------------------------------------------------------------------- 


% The original template provides and abstractseparate environment, if your institution requires them to be separate. I think it's easier to print the abstract from the complete report by restricting printing to the relevant page.
% \begin{abstractseparate}
%    
% Abstract -------------------------------------------------------------


%\begin{abstractslong}    %uncommenting this line, gives a different abstract heading
\begin{abstracts}        %this creates the heading for the abstract page

\par Indico is a web application to schedule and organise events, from simple lectures to complex meetings, workshops and conferences with sessions and contributions. The tool also includes advanced user delegation mechanism, paper reviewing, archiving of event materials. More tailored features such as room booking and collaboration are provided via plug-ins.
\\
\par Indico is heavily used at CERN and in more than 10 years, very different features are added according to the needs of increasing number of users. However, data store(ZODB) have never changed. ZODB has been very important for Indico to have been Indico but it's a bottleneck now in terms of scalability and elapsed time in feature development.
\\
\par Indico demands more scalable backend to serve increasing number of events with fancy features. The aim of the project is to replace ZODB with a highly scalable data layer which will enable global Indico, bringing $\sim$120 different Indico servers around the world together. First part of project involves the survey of different technologies according to the needs of Indico and second part is composed of preparation of transition environment with a prototype.
\\
\\
\textbf{Keywords}: SQL, NoSQL, Python, Web Development

\end{abstracts}

%\end{abstractslong}


% ---------------------------------------------------------------------- 

% \end{abstractseparate}

\frontmatter

%: ----------------------- contents ------------------------

\setcounter{secnumdepth}{3} % organisational level that receives a numbers
\setcounter{tocdepth}{3}    % print table of contents for level 3
\tableofcontents            % print the table of contents
% levels are: 0 - chapter, 1 - section, 2 - subsection, 3 - subsection

%: --------------------------------------------------------------
%:                  MAIN DOCUMENT SECTION
% --------------------------------------------------------------

% the main text starts here with the introduction, 1st section,...
\mainmatter

\renewcommand{\chaptername}{} % do not change


%: ----------------------- subdocuments ------------------------

% Parts of the report are included below. Rename the files as required.
% But take care that the paths match. You can also change the order of appearance by moving the include commands.


% this file is called up by thesis.tex
% content in this file will be fed into the main document

%: ----------------------- introduction file header -----------------------
\chapter{Introduction}

% the code below specifies where the figures are stored
\ifpdf
    \graphicspath{{1_introduction/figures/PNG/}{1_introduction/figures/PDF/}{1_introduction/figures/}}
\else
    \graphicspath{{1_introduction/figures/EPS/}{1_introduction/figures/}}
\fi

\section{Overview - What is Indico?} % section headings are printed smaller than chapter names

Indico (Integrated Digital Conferencing) is a web application for event management, initially developed by a joint initiative of CERN, SISSA, University of Udine, Nederlandse Organisatie voor Toegepast Natuurwetenschappelijk Onderzoek (TNO), and University of Amsterdam. Currently, main development is going on at CERN with occasional contributions from outside such as time zone support by Fermi Lab.

\par Indico events are divided into 3 types: lecture, meeting and conference as from simplest to the most complex, respectively. Lectures are one time talks but conferences provide all nice features of Indico.

\par In addition to events having a type, each event belongs to one category which enable category-driven navigation between events. Since finding interested event in possibly huge number of categories, time-driven (day/week/month) view is also supported. Even time-driven navigation isn't enough because a user, extensively uses Indico, can easily see tens of events in a daily view. To remedy this problem, personal dashboard is being developed in which approaching events are shown in sorted order for the user. Moreover, users can mark some categories as their favorites and Indico recommends categories to users by analyzing this information and the event history of users.

\par Indico manages full conference cycle starting with customization of website. Every part of conference management requires different access rights. One contributor should be able to modify her contribution but she shouldn't be able to others, for instance. Indico makes setup of access and modification rights easy. Conferences are nothing without registrants and Indico provides a powerful fully customizable wysiwyg registration form. Indico supports creating custom forms in addition to registration form such as to prepare a survey. Call for abstracts, rich text editing in Latex and Markdown, and paper reviewing are enabled by Indico. Indico is a document store as much as being metadata store but saving different kinds of files; pictures, slides, videos, etc. Indico is even capable of converting documents betweeen different formats such as automatic PDF generation from Microsoft Office slides. Everybody has her different flow and tools to manage her calendar. Indico allows different views of same event and exporting events in multiple formats such as HTML, XML, iCal, PDF, etc. Last but not least, Indico is extensible via plug-ins to integrate with other systems or to provide more localised services so that only Indico is sufficient for everything. Room booking to arrange event place or Vidyo for video conference may be given as examples of notable plug-ins.

\par Extensible architecture of Indico have enabled easy integration with different services. As time advanced, search engine (Invenio) is added to find catgories and events easily. Indico totally embraces mobile and recently mobile check-in application is developed. Video conference (Vidyo and EVO) and instant messaging (XMPP) are easy to use within Indico. Payment processing for registrations in different ways (PayPal, PostFinance, WorldPay, etc.) is possible. Location, room and booking management are made possible with a clear interface for places and schedule of events. As a result, Indico is absorbing new feature sets by integrating with other services and steadily advancing to become the de facto interface for most of the needs.

\section{Motivation - Why change?}

\par Main driving force in development of Indico was a required service to manage events as following agenda of CDS, CERN Document Server. Thus, rapid development of production ready software was a necessity. Since there were many features to be implemented, time spent in other parts of the system must have been minimized.

\par The beginning of 2000s was the time objects and object oriented programming languages have become mainstream. As a result, there was an opportunist storm to try objects in everywhere and data stores had their own lion's share. Thus, an object database as a back-end for web application was a meaningful decision at that time. However, object databases never took off.

\par ZODB (Zope Object Database) was a viewable option in 2002 for Python, main programming language of Indico. Since ZODB is an object database, it enables directly manipulating and transparently storing objects. This capability of data store eliminates a layer between objects in programming language and data in store which in turn, saves a lot of time, money and complexity in application.

\par Fast forward to 2014, ZODB is far from having medium sized community. What is saved before by using ZODB is now being paid to solve limitations of ZODB. Finally, net gain turned into negative which signified to replace ZODB with more mainstream, powerful and scalable data store which will support next decade of Indico.

\section{Approach - How change?}

\par Firstly, limitations of ZODB should be extensively studied because most of the time, leaving aging production system behind and writing it from scratch results in failure. It's developed for more than 10 years, leaving ZODB means discarding this accumulated knowledge. Moreover, unless problems are made ridiculously clear, clean solutions can't be found.

\par We will present these reasons and the ways how we tried to solve these limitations in existing system. After we become certain about that refactoring would be more cost-effective, next step is to define criterias for the replacement technology in terms of where ZODB has failed and also keeping the future of Indico in mind. 

\par In the golden era of data store, started by Google and Amazon to implement their extreme requirements, there are zillions of candidates, even barely knowing each of them is a really difficult task. Goal is to collect and classify candidates that conform to criterias as much as possible.

\par After first elimination and classification, next round is to get familiar with each one via implementing a very small subset of Indico. Implementation is important because making sense of abstract concepts is difficult to interpolate expectations. Even if feature implemented is small, it makes easier to predict the schedule and complexity.

\par When know-how of each system is acquired, tipping point is choosing one or subset of candidates because,  in big applications, each part has different characteristics and requirements which maps into suboptimal solutions by using only one data store.

\par In either case, transition phase is long and painful so it should be well-planned in terms of bugs, needs of community and release cycle. When schedule is carefully studied, the final bit is to implement the system conforming to the plan.

\section{Outline - What is Done and Next?}

\par In the following, we will give every tiny detail from the study of limitations of ZODB to new implementation with new polyglot database system. Moreover, database change involves refactoring huge body of code base so this project also aims to pay years of technical debt and to complete what we wish we had had in the beginning such as modular Indico. In conclusion section, how important steps are taken to modernise Indico code base and to prepare transition of Indico to Python 3 can be seen in addition to getting scalable by new back-end.

\par Even if important steps are accomplished, that's not enough because software is very volatile and requirements and technologies are quickly changing. While keeping basic principles like DRY and KISS, extending data system for specific cases will help for better scalability and facilitate the development of certain features such as statistics and social features.

	  
% this file is called up by thesis.tex
% content in this file will be fed into the main document

\chapter{Background and Related Work} % top level followed by section, subsection

\par In this chapter, we first look at some topological aspects of today's OSN, in the context of their impact on the structure of the underlying P2P infrastructure. We then focus on specific ways of offering such infrastructure, and focus on the SpiderCast protocol. We then survey the related work on building P2POSN, discuss their approaches, and point out their connection to our own work, when applicable. Finally, we present some insights about the global characteristics of the social graphs and P2P network graphs, which further motivate our work.  
          

\section{Online Social Networks}

\par The inherent characteristics of OSNs, as they were implemented in the currently available client-server architectures, should be considered when designing a P2P platform able to sustain an OSN with a large userbase. They have been comprehensively studied following (a) a purely \textit{topological view} \cite{Mislove, kumar}, focusing, for instance, on the structural position of their nodes, on their level of segmentation or connectivity, on the structural mechanisms that describe their evolution, etc; and (b) a \textit{semantic view} \cite{twitter, java}, which brings together structural characteristics and user data patterns (e.g. what type of information has a tendency to be further shared/retweeted and by what category of users, which are the most popular topics or users). All these characteristics may be used not only to help design the underlying architecture of OSN, but also to enhance P2P systems, recommendation systems, location based systems, trust computation and so on.

\par In the context of this work, we will look at the former category of studies, because we are mainly interested to see how the topology of the P2P platform impacts the performance at the OSN level (e.g. a user may not want to keep links to any other users and/or to have no links to his friends). For example, the first OSN is considered to be created by the users who exchanged email messages, as a social graph was formed in this way. Today, popular OSNs, such as Facebook \cite{web1}, Twitter \cite{web2}, LinkedIn \cite{web3} and Flicker \cite{web4}, rely on such graphs to perform basic operations like sharing, organization or data location \cite{Mislove}. 

\par It was observed that, for any OSN, the social graph is formed from three different types of groups (i.e. sets of nodes) \cite{kumar}: (a) \textit{zero-degree nodes} - usually not having an active participation within the network, (b) a \textit{giant connected component} - densely connected social network core, containing a large number of users and having at least one path between any pair of nodes (typically containing high-degree and highly active nodes), and (c) \textit{isolated communities} - small sets of nodes who interact only with each other. Note that in time the giant connected component tends to gather the vast majority of users \cite{Mislove,mascolo}, up to 99\% of nodes (as the isolated communities merge with it \cite{kumar}). This property holds for all the networks used for our system evaluation, being also preserved with high probability by the obtained overlay, as we will see in Chapter~4. We will further analyze all these characteristics and more, making also a short comparison with P2P networks characteristics, in Section 2.4.


%There is an extremely large number of publications that analysis today's popular online social networks phenomenon and their characteristics such those previously enumerated \cite{Mislove, facebook, social, twitter}. Throughout this work we will consider many of these studies when making design decisions and designing relevant and realistic test cases. Thus, in the following fifth section of this chapter we will explain in more details those relevant for our work. 



\section{P2P Networks} 

\par This thesis argues about the feasibility of a P2P system that aims to minimize as much as possible the users' tradeoffs when moving away from a social network service based on the client/server paradigm. Thus, when architecting a P2P-based OSN, we also need to deal with P2P-specific aspects, such as data availability, updates, network topology, search, data locality, etc. For this work, we mainly focus on obtaining data availability, scalability, and inexpensive communication. In this section we will point out only aspects related with this goal.

%The requirements for architecting a P2P-based OSN are related both to P2P-specific issues, as well as OSN characteristics. On one side, we need to take care of data storage and updates, network topology, search and data locality, while on the other we should face the important volume of social data typical for any popular OSN. Furthermore, when having nodes with hundreds or maybe thousands of friends in the social network, it is very costly to maintain connections with each of them in order to propagate updates (wall posts, pictures uploading, etc.).

\par The P2P model was made popular mostly by file-sharing and content distribution applications such as BitTorrent\cite{p2p1}, eDonkey\cite{p2p4}, Napster\cite{p2p2}, Gnutella\cite{p2p3}, etc. Apart from this, it is also used for real time communication between users. The nodes in a P2P network form overlays which can be classified as structured and unstructured. Even if unstructured overlays are often relatively simple, their search operations tend to be inefficient. On the other hand, structured P2P overlays usually use distributed hash tables (DHT) to perform directed searches, which makes lookups more efficient in locating data \cite{search2006, search2007}. Thus, for search performance, a structured overlay may be preferred. However, they tend to be more expensive to maintain especially under high churn rate, as their topology impose more constraints (which makes one think of the potentially large control message overhead and degraded routing performance). Unstructured overlays are arguably more resistant to churn, but it is commonly believed that they impose greater message overhead (as they generally use either flooding or random walks)\cite{Qiao,Castro}. On the other hand, when the overlay network exhibits the clustering and short path lengths of a small world network, it ensures that even message flooding approaches work well (with a low time-to-live for messages, resulting in a low message overhead) \cite{Aberer,Stutzbach}. 

\par Thus, this motivates the following question: how to propagate updates while maintaining an overlay which is scalable, churn-resistant, small diameter and with a low maintenance cost for connections. Basically, as we presume that nodes in a P2P social network will primary store the data they publish on, we need a solution for building an overlay network that allows efficient topic-based event routing in a highly dynamic environment. In this regard, the SpiderCast protocol supports publish/subscribe communication in a decentralized environment \cite{spidercast}. This protocol aims to build an overlay that is both topic-connected and has a low per-topic diameter, while requiring each node to maintain a low average number of connections. In our P2P social network case, the topics are the same with the nodes' profiles, hence the number of topics is equal with the number of nodes and each node interest contains only his friends. Thus, this protocol fits our needs since we can see the P2POSN as a publish/subscribe system in which each user subscribes to his friends. In addition, we can efficiently use the principles behind SpiderCast protocol to reduce the average node degree by manipulating the common interests (friends) between nodes for constructing a low-degree overlay in which all the nodes with a common friend form a connected component. Since the clustering coefficient is known to be high in social networks \cite{Mislove,facebook} we expect to have a large number of duplicate paths, a good premise for SpiderCast to be successful in this environment. Furthermore, the experiments on SpiderCast show that the average node degree decreases with the number of nodes \cite{spidercast}. When keeping the number of nodes constant, the increase of the number of topics moderately increases the average node degree, demonstrating also a good scalability with the number of topics.

\par In our context, we should keep in mind that the availability of any specific file is not guaranteed in a P2P network. We will see that this can become a problem for a P2P social network, as for today's OSNs this is a basic requirement. Thus, we need a cost efficient solution able to ensure data accessibility with a high probability. Towards obtaining high availability for distributed application in an environment prone to a high churn rate, a solution for self-healing active replication on top of a structured P2P overlay is proposed in \cite{semias}. Yet, each data item is associated with a unique key and its replicas are placed on the closest nodes to its key. This may not be the best approach for us as we might need to consider, for replication, node attributes such as: users diurnal behavior, number of friends, access patterns, interaction patterns, or we should provide a mean of computing the key so as to consider these attributes when computing the distance between keys, and is challenging to consider all of them at once.


\section{P2P Social Networks}

\par A P2P network can be seen as a social network where users are the nodes linked according to their friendship relations; the similarities between the two have been previously explored in \cite{case,futureosnp2p}. Such networks become as large as hundreds of millions of nodes. The authors try to identify a pertinent set of opportunities and challenges for a P2P environment in order to support the deployment of a social network. Since this identification was made, services like Diaspora \cite{webp2pd1}, considered a Facebook's young competitor and Identi.ca \cite{webp2pd2}, a solution to decentralize Twitter, have started to appear.

\par Solutions for P2P social networking have come also from the research community. Gossple, which links users based on their similarities, is a network of anonymous social acquaintances that uses a gossip protocol for updates and communication \cite{gossple}. Gossple provides a way to meaningfully associate and exchange information between users' profiles, without requiring peers to have a previous relation. This is an interesting approach as one of the reasons for belonging to a particular social network may be to meet people. Thus, a user may seek to meet people with similar interests, passions, background, problems, professions and hobbies, shifting the key focus on finding like-minded people. Basically, the nodes periodically gossip data about their interest profiles and based on this interest compute the distances between them and other nodes and create links. However, our work differs in a number of aspects as we do not look into preserving users' anonymity and to creating links based on a defined interest metric. 

\par An approach closely related with ours is PeerSoN, which aims to overcome the OSNs limitations by looking to find solutions to the privacy issues (i.e. encryption, access control and decentralization) and permitting direct exchange of data between users even without Internet connectivity, while preserving OSNs features (e.g. data availability and accessibility) \cite{peerson}. For ensuring that the private information is visible only by the right users, PeerSoN does not assume any kind of trust relationship between peers, but provides access control by using encryption and key management. While similar solutions could be adapted to the work presented in this thesis, we believe that we first need to define the design of a scalable P2POSN underlying infrastructure. 

\par LifeSocial.KOM is a P2P-based platform for secure online social networks which claims to provide the functionality of common online social networks in a totally distributed and secure manner \cite{kom}. It uses FreePastry for interconnecting the participating nodes and PAST for reliable, replicated data storage. Alas, it was tested at a very small-scale only. 

\par In \cite{porkut} a privacy aware decentralized OSN called Porkut is introduced. The idea behind it is to exploit trust relationships in the social network for decentralized storage of OSN profiles and to address availability and accessibility issues by considering users' geographical locations and online time statistics, while using k-anonymity techniques for ensuring a privacy preserving indexing. Yet, they store the data in plain text relying on social pressure and monitoring for data integrity and privacy. A solution, for providing reliable storage, is to offer proper incentives to nodes, based on existing social relationships, in order to make them to cooperate for storing data and to remain in the system \cite{f2f}. However, this approach is suitable mainly for data intensive applications that do not require global visibility or global sharing. 

\par There are also customized solutions such as SCOPE which is a prototype for spontaneous P2P social networking \cite{scope}. It provides the distributed database and lookup services deployed on DHT technology so as to support social networking in local areas. Finally, the Maze P2P network integrates both a friends' social network and a download network. The similarity between nodes' interest is revealed by the social network relationship within the Maze network \cite{maze}. Thus, Maze is improves the content search in the download network by using the ``small world'' property \cite{small} of the friends' social network as it enables small search path length.

 
\section{Brief Analysis of Social and P2P Networks Graphs}

\par When trying to determine if a shift from a client/server to a P2P infrastructure is feasible, all the characteristics of today's OSNs such as clustering coefficient, path length, access patterns, interaction patterns, etc., need to be considered so as to make a realistic feasibility study.

 \begin{table}[t]
    \begin{tabular}{ | p{4.6cm} | p{4.0cm} | p{4.0cm} |}
    \hline
    \textbf{Network} & \textbf{Average path length} & \textbf{Type (OSN or P2P)}\\ \hline
    Facebook\cite{facebook} & 4.8 & OSN  \\ \hline
    Flicker\cite{Mislove} & 5.67 & OSN  \\ \hline
    Twitter\cite{twitter} & 4.12 & OSN \\ \hline
    Maze friends network\cite{maze} & 7.69  & P2P  \\ \hline
    Maze download network\cite{maze} & 9.17 & P2P \\ \hline
    \end{tabular}
    \caption{Average path lengths for OSNs and P2P networks}
    \label{small_tab}
\end{table}

\par Social graphs exhibits the ``small world'' property which practically measures to what extent a network can spread information quickly and widely \cite{small}. It is obtained with a small average path length and a high clustering coefficient. We observe that while ONSs have their average path length lower than that computed for world population [6.6] \cite{six}, the average path length for P2P networks is higher. The average path length values for some of the most popular social networks and some P2P networks are presented in Table \ref{small_tab}. The values for average path lengths and clustering coefficients obtained for all these graphs (so as for any ``small world'' network) compared with those obtained for a corresponding random network, with the same number of nodes and average degree per node, have an average path length close to the random network, while the clustering coefficient is much larger \cite{small}. The data in these papers suggests that even if for P2P networks the path length is slightly bigger and the clustering coefficient slightly lower, they seem to have similar topological structure. More, with redundancy and some solution for information availability P2P social networks can reach ``six degrees of separation" and in a P2P network we can take advantage of this short average path length. For example, flooding with relatively small time to live would cover most of the graph. In addition, if this is correlated with a large clustering coefficient, interest based clusters of peer can be identified. These features can influence the performance for operations such as information dissemination, friends search, etc., in a P2P social network.
 
 
 
 %\par The first observation is that when the average number of friends per user is increasing the average path length has a bigger probability to decrease, because the probability of intersection between groups of friends is increasing. Second, when there are asymmetric relations as in Twitter, where users follow other users to obtain information, the path length will also decrease \cite{twitter}. 

%\par In fact the small values of average path length of ONSs are almost certainly obtained because of the unnatural characteristics of these networks in comparison with real life relation. For the same reason a large clustering coefficient is improbably to be obtained with real life social relations. In addition, in the case of Maze \cite{maze}, which use peers social relations, can be observed that the obtained numbers for path length are bigger. 

\par Some very interesting findings show that in unstructured Gnutella-like P2P overlays: a) the long-lived peers form a densely connected and stable core, which leads to stable and high connectivity among peers even in the presence of high churn and b) the network dynamics lead to a core overlay that exhibits �onion-like� biased connectivity (peers form ``layers'' depending on their uptime - peers with shorter uptime form a layer by forming connections with each other and with peers with higher uptime) \cite{Stutzbach}. This is quite similar with how the OSNs evolves in time: old users have a tendency of having larger friendbase and to move closer to the social network core and to also form a densely connected core \cite{kumar}.

\par  Considering only the ``static'' social graph does not suffice. A proof of the fact that social links are not valid indicators of real users' interactions, for the biggest OSN Facebook, is given in \cite{facebook}. There is a high skew distribution of users interaction, for instance less than 1\% of Facebook users interact with more than 50\% of their friends. Consequently, the social links should include data about users' interactions. For 50\% of users, 70\% of interaction comes from 7\% of friends and 100\% of interaction comes from 20\% of users. Basically, this also mean that the interaction graph can eliminate useless links in the ``static'' social graph, as the results correspond with theoretical limits on human social cognition\footnote[1]{ This limit is given by Dunbar's number (150) which represents the maximum number of stable social relations that one can maintain.} \cite{small}. These invalid links are the reason why, for instance, the OSNs exhibit lower values for path length than those obtain for world population. 99\% of Facebook users interact only with less than 150 friends. While the interaction links weight (i.e. the number of interactions between to users) could be used to further enhance our P2POSN system design by giving priority to stronger links (i.e. links with higher weight), this work focus on keeping all node's friends profiles accessible, regardless the interaction patterns. %In Chapter 4 we will discus and compare results obtained on both ``static'' and interaction social graphs.

\par However, while users' visible interaction links provide clues on how to design a social platform, the majority of user interactions in OSNs are latent interactions (e.g. browsing a user profile) \cite{latent}. This study is made on Renren, the largest OSN in China, on 42 million users, 1.66 billion social links and detailed histories of profile visits over a period of 90 days for more than 61,000 users. It analysis the latent interaction graph which basically models users' browsing behavior. In addition, it shows that the latent interaction graph has a clustering coefficient (0.03) lower than those obtained for social graph (0.18) and visible interaction graph (0.05), while in terms of connectivity it falls between them \cite{latent}. These characteristics could be relevant for us mainly for data location and replication. They allow the weighting of all the social links, in order to give priority to some particular links as well. Yet, this extension is beyond the scope of our work.

 






% ---------------------------------------------------------------------------
% ----------------------- end of thesis sub-document ------------------------
% ---------------------------------------------------------------------------					
% this file is called up by thesis.tex
% content in this file will be fed into the main document

\chapter{Survey of Candidates} % top level followed by section, subsection

\par In this chapter, we try to look into many databases as much as possible to compare strong and weak places and to find the best possible fit.

\section{Main Database Types}

Database systems are mainly divided into 6 categories:

\begin{enumerate}
  \item Object-Oriented
  \item Relational
  \item Column
  \item Document-Oriented
  \item Key-Value
  \item Graph
\end{enumerate}

Each database type and possible candidates from each type are explained further.

\section{Object-Oriented Databases}

Object-Oriented databases bring capabilities of object-oriented programming languages into database world. Objects in programming languages can directly be stored, modified or replicated because these databases use the same object models as the corresponding programming languages. Unlike relational databases, there is no layer for conversion of objects back and forth. However, that requires database to be tightly integrated with programming language so that it provides more transparent storage.

Object databases have been around since object-oriented programming has become popular by Small Talk at 1970s. Although they have a long history, their main showcase started with Java invasion of software world. Even if the place of Java and object-oriented programming in software world are obvious, object-oriented databases have never been able to become mainstream like relational databases. As a result, there are no very successful products which leaves us with small number of choices to consider. ZODB and WakandaDB are the main viable systems.

\subsection{WakandaDB}

\vspace{-1.15cm} \hspace{4.1cm} \includegraphics[scale=0.8]{3/figures/wakandadb.png}

WakandaDB is very new, only half year passed since first public stable release at the time of writing. It pushes the motto of \textit{JavaScript is at everywhere} which is a new wave pioneered by \textit{Node.js}.

WakandaDB is an ambitious project because developers are trying hard to create a suite of tools that will work together as a charm and will definitely be needed in the life cycle of an application. Currently, WakandaDB comes with:
\begin{itemize}
  \item \textit{Wakanda Studio}: IDE with full featured debugger
  \item \textit{Model Designer}: Classes and schema are easily created
  \item \textit{GUI Designer}: Makes easier creating visual interfaces
  \item \textit{Web Application Framework}: Plug and play framework to use WakandaDB with REST API
\end{itemize}

It is much more than a simple database product. This may be seen as an advantage since these tools save developers from tons of problems and details but it isn't actually perfect in terms of Indico. It forces its own custom stack but chosen database should coexist with other parts of Indico. Database change is already a big project in its own so tool chain and framework replacement aren't expected. Furthermore, its JavaScript push isn't nicely fits into Python stack of Indico but we should also note that JavaScript percentage in the code base of Indico has been increasing clearly with interface renovations.

If we were starting a web project now, WakandaDB seems a promising option with tooling and feature of totally embracing web. However, it is a bit dangerous even that situation since it's too early to play on WakandaDB as seen from lack of success stories and big deployments.

\begin{table}[h]
  \centering
  \caption{WakandaDB Object-Oriented Database}
  \renewcommand{\arraystretch}{1.5}
  \begin{tabular}{| >{\centering\bfseries}m{1in} | >{\centering\arraybackslash}m{4.5in} |}
	\hline
    \textbf{Criteria} & \textbf{Feature} \\
	\hline
    Availability &
    Dual license, AGPLv3 and commercial. \\ \hline

    Scalability \\ Replication &
    It's in infancy of development and stabilization.
    There aren't enough successful and big deployments.
    Since its target is specific, there are no benchmarks
    to compare its performance characteristics. \\ \hline
    
    Ease of Use Development &
    JavaScript is the only first class supported language.
    Server supports REST API so in theory every language
    can be leveraged via HTTP but Python isn't fully ready
    for the time being. \\ \hline

    Transactions Consistency &
    Transactions, even nested, are fully
    supported with ACID guarantees. \\ \hline

    Community Momentum &
    New project so it's normal that community
    is just starting to get together but as
    seen from version control history, development
    isn't going on fast enough. Using only one language
    in every part of stack is a huge time-saver and
    JavaScript is living its golden age so the choice of
    \textit{JavaScript at Everywhere} makes sense but
    results aren't obvious yet. \\ \hline

    Cost \\ Exit Strategy &
    Starting to whole tool chain would be huge but
    if only server component is utilized, cost can be reduced.
    Even using only server makes requires extensive JavaScript
    knowledge and leaving accumulated Python know-how.
    Exit would be easier compared to ZODB because there is an
    official MySQL connector which can export all data to be
    used in MySQL with little effort. \\ \hline

    Score &
    \rpt[3]{\FiveStarOpen}\rpt[3]{\FiveStar} \\
    \hline
  \end{tabular}
  \label{wakandadb}
\end{table}

\subsection{ZODB}

The Zope Object Database is an object-oriented database to transparently persist Python objects.
In the beginning, it was a part of web application framework but later it is extracted to be used independently.

ZODB is a directed graph of Python objects where its starting vertex is a dictionary called \textit{root}.
Objects must be attached to \textit{root} or one of its branches to be persisted.
Thus, persisted objects can be accessed by starting from root and following links to children of \textit{root}.

ZODB has powerful features in addition to being transparent storage:
\begin{enumerate}
  \item ACID transactions
  \item Version control for objects which causes packing problem unless properly automated. Each version of objects are stored which dramatically increases database size. In advanced usage, old version of objects may be needed but Indico doesn't require so old objects are deleted daily(db is \textit{packed}). This may be a serious problem for a system heavily in use with limited disk space. Unfortunately, Indico fits into this system class.
  \item Pluggable storage
  \begin{enumerate}
    \item File: Only one file on file system and it's in use Indico.
    \item Network aka. ZEO (Zope Enterprise Objects): It allows multiple client processes to access database concurrently which enables easy scaling but ZEO server itself becomes bottleneck and a single point of failure.
    \item Relational Database: Brings independence from one particular product at the expense of extra complexity and overhead.
  \end{enumerate}
  \item Client caching: Just a remedy for the lack of caching in server
  \item MVCC: Reads aren't blocked while only one write is allowed.
  \item Replication via \textit{ZRS} (Zope Replication Services): It is recently open sourced (May 2013). It was totally out-of-picture before due to its ridiculous price. ZRS enables single point of failure by providing hot-backups for writes and load-balancer for reads.
\end{enumerate}

It has been around since late 90s so it is reasonable to call ZODB as the most mature Python object database in the wild. It powers many successful products in which Indico exists.

One interesting drawback of ZODB being very tightly coupled with code base is that each object is stored by their full class name to disk. That prevents moving classes around without a migration step. \textit{Makac}, Make a Conference, name was chosen for Indico at first development stage. As a result, code base contains a lot of code under \textit{Makac} package or prefixed by \textit{Makac}. Indico name is the final which means Indico should replace \textit{Makac}. It wasn't easy like just copying files but database change makes this required adaptation easier. Therefore, what we wished for at start seems possible such as modular Indico.

\begin{table}[!ht]
  \centering
  \caption{ZODB Object-Oriented Database}
  \renewcommand{\arraystretch}{1.5}
  \begin{tabular}{| >{\centering\bfseries}m{1in} | >{\centering\arraybackslash}m{4.5in} |}
	\hline
    \textbf{Criteria} & \textbf{Feature} \\
	\hline
    Availability &
    Zope Public License \\ \hline
    Scalability &
    Although there are solutions like ZEO and ZRS (recently), it's limited for the Indico use case which involves multiple big entities and their relationships such as conferences, avatars, reservations. ZODB is successful for big entities but their relationships is where ZODB performs poorly. \\ \hline
    Ease of Use Development &
    ZODB is more like a library than standalone program which makes setup easy.
    It's tightly integrated to code which is both an advantage and a disadvantage.
    ZODB puts or retrieves objects without interfering with you but development should be done as it expects.
    The result is that ZODB is barely seen while you're developing with it but when you would like to change,
    you can't because it's in any tiny part of code \\ \hline
    Transactions Consistency & ACID \\ \hline
    Community Momentum & Even if it's most well-known and mature object database in Python world, it's a niche project by being object database. Documentation and surrounding tool chain are lacking despite its long history. In overall, the company behind of ZODB is in the decline overall. \\ \hline
    Cost \\ Exit Strategy & Starting to use ZODB is easy by simple setup and heavy integration with Python. These are main reasons why it's chosen but not interesting any more like exit and exit isn't easy again due to same reasons; namely, tightly-coupled code base. \\ \hline
    Score & \rpt[3]{\FiveStarOpen}\rpt[3]{\FiveStar} \\
    \hline
  \end{tabular}
  \label{zodb}
\end{table}

\section{Relational Databases} 

A relational database is a collection of tables of data items (\textit{row}) in a relational model. Each table has a strict schema which specifies data parts, their types and their relationships to each other. Each row has a \textit{primary key}, composed of one or multiple columns, uniquely identifies itself. Relationships can be established within a table or between tables via \textit{foreign key} which is a pointer to primary key of some table. Foreign keys provide various level for normalization of tables. Normalization is a methodology to avoid data manipulation anomalies and loss of integrity by preventing redundancy and non-atomic values in table design. After schema is normalized, less intuitive compared to non-normalized data, arbitrary queries can be executed on the schema. This ad-hoc query capability is the most important feature of relational world and nicely fits into evolving and agile software development. However, what is won by dividing data is lost when data spanning multiple tables is requested because data can only be recreated by joining tables, combining rows with referenced foreign keys, which are very costly in terms of performance. Moreover, normalization requires touching multiple tables at the same time to join data but relational database systems are fully ACID compliant with tunable guarantees and levels such as row or column level. Furthermore, with the push of big data requirements, replication is supported at default. Finally, schema design and normalization level are deadly important for sharding. Having both replication at default and good design, scalability, even huge scale, isn't a problem.

Relational databases category is the one most battle-tested for at least 20 years and well-studied for the last 40 years. As a fruit of it, there are multiple very successful deployments, huge living community with extensive know-how, powerful tools.

\subsection{PostgreSQL}

\begin{table}[!ht]
  \centering
  \caption{PostgreSQL Relational Database}
  \renewcommand{\arraystretch}{1.5}
  \begin{tabular}{| >{\centering\bfseries}m{1in} | >{\centering\arraybackslash}m{4.5in} |}
	\hline
    \textbf{Criteria} & \textbf{Feature} \\
	\hline
    Availability &
    PostgreSQL License (BSD or MIT-like) and clear roadmap and open development \\ \hline
    Scalability &
    Scalability is tested with successful deployments: Amazon, Dropbox, Heroku,
    Instagram, Reddit, Skype, NASA, Yahoo\\ \hline
    Ease of Use Development &
    Object-relational mapping (ORM) layer is needed but there is a well developed and supported library, SQLAlchemy,
    which is in production at Dropbox, Yelp, Uber, OpenStack. \\ \hline
    Transactions Consistency & ACID \\ \hline
    Community Momentum &
    One of two biggest communities in database ecosystem and it's stealing users from MySQL due to unclear future and commercialization since acquisition of Sun Microsystems, primary developer, by Oracle. Although many frameworks and PaaS providers support relational databases in general, they recommend PostgreSQL usage notably as a result of standard compliance. \\ \hline
    Cost \\ Exit Strategy & Integration of ORM layer makes code back-end agnostic unless back-end specific extensions are used. However, special features and types make code more efficient which can be even reduced by providing ORM with back-end specific package to transfer queries in the intended way. \\ \hline
    Score & \rpt[6]{\FiveStar} \\
    \hline
  \end{tabular}
  \label{postgresql}
\end{table}

\subsection{MySQL}

MySQL is the most popular relational database by far. MySQL owes its popularity to investing into its performance by lacking some features exist in PostgreSQL. This is a long story which has a dramatic end because PostgreSQL increased its performance in recent versions while MySQL has adapted most of the lacking features. Even if gap is closing in simple queries, PostgreSQL performs better in complex queries involving sub-queries and/or multiple joins via its well-studied query optimizer.

There is a subtle detail in this comparison. PostgreSQL is an integrated database, one block of code base but MySQL is composed of two layers, SQL (parser, optimizer, etc.) and storage layer (responsible for actual data storage, modification, etc.). Performance characteristics and features, transaction guarantees, of storage layers are highly different. Thus, carefully specifying storage is important for a healthy comparison. In that respect, MySQL supports multiple storage engines; namely, InnoDB, BerkeleyDB, TokuDB and MyISAM, etc. but in terms of ACID requirements of Indico, development activity, and better comparison to PostgreSQL, InnoDB is our main concern.

\begin{table}[H]
  \centering
  \caption{MySQL with InnoDB vs PostgreSQL Comparison}
  \renewcommand{\arraystretch}{1.5}
  \begin{tabular}{| >{\centering\bfseries}m{1in} | >{\centering\arraybackslash}m{4.5in} |}
	\hline
    \textbf{Winner} & \textbf{Features} \\
	\hline
    $=$ & MySQL (InnoDB) is very similar to PostgreSQL in terms of stored procedures, triggers, indexing and replication. \\ \hline
    PostgreSQL & Richer set of data types, better sub-query optimization, fully customizable default values and better constraints for foreign keys and cascades. Standard compliance. PostgreSQL has a clearer roadmap and a more permissive license compared to MySQL because after Oracle acquisition, MySQL has become open-source product but lost its open source development culture. As a solution, some concerning old MySQL developers forked project but this results in multiple MySQL-like products in the market such as MariaDB, Drizzle, etc. Moreover, with the help of layered architecture, different companies are providing similar but different products with reusing SQL layer and plugging their custom storage layer such as Percona Server, TokuDB, etc. \\ \hline
    MySQL & Advanced concurrency primitives like REPLACE - check and set atomically, and richer set options for horizontal partitioning. \\ \hline    
  \end{tabular}
  \label{mysql}
\end{table}

In short, MySQL is a very successful relational database that has a flexible design and is targeting performance at the expense of diverging standard and lacking features. However, new owner Oracle, the company also has the biggest share in commercial database world, damages MySQL.

\begin{table}[!ht]
  \centering
  \caption{MySQL Relational Database}
  \renewcommand{\arraystretch}{1.5}
  \begin{tabular}{| >{\centering\bfseries}m{1in} | >{\centering\arraybackslash}m{4.5in} |}
	\hline
    \textbf{Criteria} & \textbf{Feature} \\
	\hline
    Availability &
    Dual license (GPLv2 and commercial) and what Oracle is trying to do with MySQL isn't clear. \\ \hline
    Scalability &
    Scalability wouldn't be problem since it serves well even in the scale of Facebook, Google and Twitter.
    \\ \hline
    Ease of Use Development &
    ORM layer is needed but code targeting for PostgreSQL with SQLAlchemy can be adopted for MySQL unless richer types of PostgreSQL are leveraged extensively. \\ \hline
    Transactions Consistency &
    ACID in InnoDB \\ \hline
    Community Momentum &
	According to various rankings, MySQL is the most popular open source database with huge community. Oracle damaged its reputation for openness. Forks emerged and followed closely with upstream until now but after stabilization of long waited 5.6 version, first version in NoSQL storm, forks started to diverging. MariaDB versions were one-to-one mappable with MySQL versions until 5.5 but with 5.6 version, they even chose to change version scheme to start with 10.0, for instance.
	\\ \hline
    Cost \\ Exit Strategy &
    ORM layer makes a relational database replacement easy so what is said for PostgreSQL is also valid for MySQL. Their difference are getting subtle and unimportant. Moreover, MySQL has forks for replacement without doing any change if they continue following trunk but they started to signalling detachment.
    \\ \hline
    Score & \rpt[1]{\FiveStarOpen}\rpt[5]{\FiveStar} \\
    \hline
  \end{tabular}
  \label{mysql}
\end{table}

\subsection{SQLite}

SQLite is an embedded relational database management system. In lay-man terms, it is a tiny C library that can be dynamically linked. Unlike MySQL or PostgreSQL, it's not a separate process that is waiting for requests. On the contrary, it is a part of the application and can be pictures as an independent module within application and can be used as function calls like other methods.

All information related to database is stored into a platform agnostic file. This is similar to file storage of ZODB and by copying that file, database can easily moved to wherever it is required.

Writes lock this file which means writes are executed sequentially but reads don't require that lock which enables concurrent reads. If multiple writes happen concurrently, only one of them can gather the lock and others will fail with a specific error code. With these execution characteristics, it's simpler than PostgreSQL and MySQL.

It's self-contained, server-less (no daemon for waiting client connections), zero-configuration relational database which makes it a universal solution for small needs but it's not capable of serving high load of production. As a result of this ease of use, it is preferred by many high caliber products such as Android and Firefox. In the context of Indico, it makes perfect sense for testing. However, many features provided by PostgreSQL and MySQL are missing in SQLite so while keeping MySQL or PostgreSQL as main data back-end, using SQLite even for tests brings complexity.

\begin{table}[!ht]
  \centering
  \caption{SQLite Relational Database}
  \renewcommand{\arraystretch}{1.5}
  \begin{tabular}{| >{\centering\bfseries}m{1in} | >{\centering\arraybackslash}m{4.5in} |}
	\hline
    \textbf{Criteria} & \textbf{Feature} \\
	\hline
    Availability &
    Public Domain, Universal and even comes with Python. \\ \hline
    Scalability &
    A big problem because it is designed to be an embedded light database system. It may be thought for only testing since no configuration is required and database can totally be kept in memory for running tests faster.
    \\ \hline
    Ease of Use Development &
    ORM layer is needed as being a relational database system and it is supported by SQLAlchemy. \\ \hline
    Transactions Consistency &
    ACID
    \\ \hline
    Community Momentum &
    Due to its popularity (Android, Chrome, Firefox, Opera, Skype, ThunderBird, etc.), community is huge and it is well tested and documented (more than 2 million tests are run for each release).
    \\ \hline
    Cost \\ Exit Strategy &
    ORM layer provides good abstraction but lacking features raise question marks. \\ \hline
    Score & \rpt[1]{\FiveStarOpen}\rpt[5]{\FiveStar} \\
    \hline
  \end{tabular}
  \label{sqlite}
\end{table}

\section{Column Oriented Databases}

\subsection{Cassandra}

Apache Cassandra is an open source distributed (mainly) column oriented key-value store achieving high performance and availability. According to CAP theorem, a distributed system can't provide consistency, availability and partition tolerance at the same time. As a distributed database, Cassandra chooses to provide availability and partition tolerance (aka. distributed) with no single point of failure by assigning same role to each node in the cluster. In addition, Cassandra supports master-less asynchronous inter-cluster replication for higher availability guarantees. Meantime, third dimension, consistency, can also be adjusted according to the needs of applications. Since there is no difference between nodes, performance linearly scales as much as new nodes are added into cluster without interrupting existing nodes.

Due to its data model, it may actually be seen as a hybrid between key-value stores (inspired by Amazon Dynamo) and column stores (inspired by Google BigTable) because each row aka. record (a variable number of columns) is identified with unique key which is distributed in the cluster by random partitioner or order preserving partitioner so that similar keys are closer to each other. Partitioning according to this key means assigning rows onto physical nodes. Primary keys can be composed of multiple columns and after rows are partitioned by row id, rows are clustered by the remaining columns.

Row partitioning requires a unique id to identify and to keep track of records. In relational world:
\begin{itemize}
  \item Single server: unique id is just an auto incrementing counter.
  \item Multiple servers: application logic (not a solution, escaping solving problem in database), 3-rd party service or ticket servers. Latter options bring extra components and they may also be a single point of failure and write bottlenecks.
\end{itemize}

As seen, solving this problem in a distributed environment is hard. Relational options aren't usually identical but Cassandra nodes are identical so this problem should be solved while keeping this architecture. Permitting each node generate ids may result in inconsistencies and lock is needed to prevent . However, lock means waiting and performance degradation where the main objective of Cassandra is high throughput.

Cassandra tries to solve it not much differently than relational counterparts:
\begin{itemize}
  \item Generates id from data
  \item Uses Universal Unique ID (UUID), easily produced by the client
\end{itemize}

Structure of tables are pre-defined like relational databases but unlike RDBMS, they can be modified on the fly without any downtime (no blocking for queries). Number of columns can vary from 1 column to 2 billion columns where each column has a name, value, timestamp and TTL (for expiration). Each row doesn't need to contain exact set of columns.

In RDBMS, fully normalization is recommended to prevent update anomalies. In Cassandra, there are more higher level abstractions to store 1 to many relations in a column; namely, set, list and map and Cassandra doesn't support joins and sub-queries except Hadoop tasks (Pig and Hive are also supported) so de-normalization is encouraged via manage logical relations by using these abstractions.

\begin{table}[!ht]
  \centering
  \caption{Cassandra Column-Oriented Database}
  \renewcommand{\arraystretch}{1.5}
  \begin{tabular}{| >{\centering\bfseries}m{1in} | >{\centering\arraybackslash}m{4.5in} |}
	\hline
    \textbf{Criteria} & \textbf{Feature} \\
	\hline
    Availability &
	Apache Software License 2.0 and Cassandra is one of the most influential top level projects of Apache Software Foundation. \\ \hline
    Scalability &
    It's a clear winner in terms of throughput with increasing number of nodes by supporting clusters spanning multiple data centers with master-less replication. Especially, it's still perform well under heavy write load. Other databases requires locks for writes but Cassandra doesn't.
    \\ \hline
    Ease of Use Development &
    It's designed to handle very massive datasets and shines under a huge deployment with thousands nodes. Highly distributed nature requires some excellence to leverage it. However, its data model is highly similar to relational counterparts and supports collections, which are higher level primitives. For instance, one room can have multiple equipments. In relational, there is a need for a table that maps rooms and equipments since this is a many to many relationship. In Cassandra, this can easily be handled by \textit{set} collection. Moreover, de-normalization with collections is encouraged in Cassandra since there is no join. Therefore, converting deeply nested objects of Indico into Cassandra data model is easier than fully normalized relational tables.
    \\ \hline
    Transactions Consistency &
    There was no support for transactions but there are good efforts to bring and as a result, latest version supports lightweight transactions, which are far from ACID because it's only very specific case of generic transactions. Transactions enable to do multiple operations on multiple tables without any interference but lightweight transactions (very misleading naming) only support doing two things on the same row by a compare and swap operation.
    \\ \hline
    Community Momentum &
    Cassandra is the most popular column store.
    Companies that have high write loads like Facebook, Reddit and Twitter are users of Cassandra.
    \\ \hline
    Cost \\ Exit Strategy &
    Cassandra uses its own query language, \textit{CQL}, which isn't a standard SQL but pretty similar to SQL so making an intra-class change is much harder than other types of database systems. However, there are some projects recently to use Cassandra as a back-end such as MySQL/MariaDB and Titan, a graph database. Thus, transition to these products may require less effort.
    \\ \hline
    Score & \rpt[2]{\FiveStarOpen}\rpt[4]{\FiveStar} \\
    \hline
  \end{tabular}
  \label{cassandra}
\end{table}

\subsection{HBase}

Apache HBase is open source clone of Google's BigTable. It's fault tolerant column oriented database with compression and version control for data. HBase is a CP system according to CAP Theorem.

Records are accessed, sorted and distributed by row key. A master node keeps tracks of assignments to slaves, \textit{region} servers because records are sorted and an assignment is a continuous range so clients can easily learn where to look for data by knowing lower and upper bounds of regions.

It's easily thought that HBase isn't scalable due to the existence of a master node. On the contrary, system is actually scalable because master node isn't involved with data requests. On the other hand, reassignments of regions according to size (split or combine) and table operations require master to be available. Thus, if master is down, data can be served by region servers because clients (can) cache the boundaries of region servers. However, a cold client can't learn assignments because assignments are kept in meta table stored in master which isn't up now.

HBase runs on top of Hadoop Distributed File System (HDFS). Adaptation phase of HBase becomes more complex due to HDFS setup and configuration but when it's there, replication of data comes free with HDFS. HBase is designed to crunch very very big data, peta-bytes of data. That's why HDFS requirement seems reasonable but  if we think about Indico, HBase is much more complex solution than enough. It's basically killing a mosquito with an atom bomb. It's in production in data-rich parts of high load systems such as Facebook, Twitter and Yahoo. 
% Double check me !!!!!!
Google has its own version, BigTable and it was very successful to be universal back-end at Google. Google is working on the next generation of BigTable called Spanner which is storage engine of Drive, GMail, Groups and Maps.

HBase doesn't support transactions but atomic operations are supported in the row level. ACID like guarantees provided by relational databases is achievable via de-normalization. However, nesting data requires more or less to know access patterns. Otherwise, ad-hoc queries would be a necessary but they aren't possible to lack of transactions. Transaction problem is solved in the design phase with well-planned row key and nesting data into one row.

\begin{table}[!ht]
  \centering
  \caption{HBase Column-Oriented Database}
  \renewcommand{\arraystretch}{1.5}
  \begin{tabular}{| >{\centering\bfseries}m{1in} | >{\centering\arraybackslash}m{4.5in} |}
	\hline
    \textbf{Criteria} & \textbf{Feature} \\
	\hline
    Availability &
    HBase may be the biggest kahuna of the systems that can handle massive data.
    HBase is also a top level project of Apache Software Foundation, licensed under Apache License 2.0 and it's a part of Hadoop ecosystem so runs on top of HDFS.
    \\ \hline
    Scalability &
    It follows the design of Google's BigTable and highly scalable.
    Like TaskTracker and JobTracker in Hadoop, or NameNode and DataNode in HDFS, HBase also has a master and slaves.
    Master handles administrative operations for auto-sharding and making assignments for slaves, region servers.
    This mapping is stored in META table at ZooKeeper node.
    Client need to know where each region is stored but they don't need to talk to master every time when they need data, instead clients can keep a cache and can directly connect to slaves.
    If there is a reassignment in regions, clients will get an exception to invalidate cache and will be required to relearn region assignment.
    Therefore, even if it seems it doesn't scale linearly due to a master-slave architecture, it is actually scales because master is only there to coordinate and master doesn't involve in data requests.
    However, there are some availability concerns related to master.
    If master is down, administrative operations can't be accomplished such as table creation/update but slaves can continue serving data operations even if master is down.
    \\ \hline
    Ease of Use Development &
    Since it's a part of Hadoop, it requires Hadoop machinery in place.
    Setup process of HBase is the most involving of all candidates.
    However, if the system is installed, it can easily handle peta-bytes of data and provides a giant table view of data to clients irrespective of where data is stored.
    \\ \hline
    Transactions Consistency &
    HBase isn't an ACID compliant database but it provides some of properties such as atomic mutation in the row level even if mutation includes multiple column families but that isn't enough in the context of Indico because Indico has a pretty complex schema and whole schema can't be nested so that one magic row key will enable atomic operations.
    \\ \hline
    Community Momentum &
    HBase is a successful product and has a lot of well-known users but migrations seem to be cumbersome.
    That's why users that don't really have (really) big data problems in the scale of HBase (tens of peta-bytes) are going away from HBase in the favor of simpler solutions.
    \\ \hline
    Cost \\ Exit Strategy &
    Entering and exiting from HBase are costly because it lives within Hadoop ecosystem and there is no direct replacement as powerful as HBase for very big data.
    \\ \hline
    Score & \rpt[2]{\FiveStarOpen}\rpt[4]{\FiveStar} \\
    \hline
  \end{tabular}
  \label{hbase}
\end{table}

\section{Document Oriented Databases}

Document-oriented databases are middle compromise between key-value store and RDBMS.
Key-value stores identify records by a key and values aren't interpreted, only seen as a blob of bytes, instead.
On the other hand RDBMS requires a strict schema even for values (columns in relational terminology).
Document databases try to provide benefits of both worlds because they don't have a strict schema for values and can also give a meaning to the values so that they can index and query by values.

\subsection{MongoDB}

MongoDB is the most popular document-oriented database.
It stores flexible JSON-style documents and provides a set of functionalities that is closer to that of RDBMS, namely a full-fledged query interface, full indexing support and in-place updates.
The schema of a MongoDB is basically non-existent - there is the concept of \textit{collections} that can be compared to relational tables, but objects can have whatever fields and values the programmer sees fit. An object is not constrained in any way just because it belongs to a particular collection.

\begin{table}[!ht]
  \centering
  \caption{MongoDB Document-Oriented Database}
  \renewcommand{\arraystretch}{1.5}
  \begin{tabular}{| >{\centering\bfseries}m{1in} | >{\centering\arraybackslash}m{4.5in} |}
	\hline
    \textbf{Criteria} & \textbf{Feature} \\
	\hline
    Availability &
    AGPLv3\\ \hline
    Scalability &
    Replication and sharding comes at default which are cornerstones of a scalable system. MongoDB supports master-slave replication via replica sets to provide redundancy and higher availability.
    Auto-sharding is possible with configuration servers, sharded meta-data holders, and query routers, \textit{mongos}.
    Moreover, it is faster compared to competitors since it loosened some constraints. However, these weak guarantees can bite the application developer if application is very strict about data lost/consistency even under high load.
    MongoDB applies exclusive read-write lock so multiple reads can happen at the same time but a write requires a exclusive lock. There were problems related to the extent of write lock like being for whole \textit{mongod} instance but it is recently improved to be for a collection which should be improved for document level.
    \\ \hline
    Ease of Use Development &
    Python is one of the officially supported languages.
    MongoDB doesn't have a schema at all and very flexible. As a result, using MongoDB is a pleasure and development with MongoDB takes less time for the same feature in RDBMS.
    \\ \hline
    Transactions Consistency &
    MongoDB gives guarantee of BASE, eventual consistency and it doesn't support transactions which is a really bad disadvantage for Indico where there are multiple inter-relations between entities. Some parts of Indico nicely fit into a document but putting everything into a document is difficult in terms of design complexity and size constraint. Then, inter-relations discourage MongoDB as main storage.
    \\ \hline
    Community Momentum &
    Without any doubt, MongoDB is the most popular NoSQL database in the wild and it is very well documented.
    \\ \hline
    Cost \\ Exit Strategy &
    Replacing MongoDB with documented oriented databases wouldn't be much difficult because all of them; CouchDB, RethinkDB, RavenDB are similar at the core.
    Moreover, documents are close to objects as in ZODB. That's why coming from object-oriented world into document databases is a smoother path to follow.
    \\ \hline
    Score & \rpt[1]{\FiveStarOpen}\rpt[5]{\FiveStar} \\
    \hline
  \end{tabular}
  \label{mongodb}
\end{table}

\subsection{CouchDB}

Apache CouchDB is a multi-master document-oriented database that is completely ready for the web.
From CAP Theorem, it's closer to AP system because it favors availability with its multi-master architecture while resolving conflicts are delegated to clients.
MongoDB vs CouchDB has caused many heated debates but this discussion actually can be reduced to whether consistency (CP) or availability (AP) is more important for the application.

Document model of CouchDB is very similar to the model of MongoDB. Documents, key-value pairs as in JSON, has a unique id to differentiate themselves.
Values can be primitive values but also more complex types like lists or associative arrays.

CouchDB doesn't support joins but can support ACID semantics in the document level via implementing MVCC not to block reads by writes.
Joins are possible via Map/Reduce tasks implemented as JavaScript functions.
This tasks are called views in CouchDB terminology and they can also be indexed and synchronized to updates to documents.

CouchDB provides eventual consistency in addition to availability and partition tolerance.
Each replica has its own copy of data, modifies it and sync bi-directional changes later.
This design brings offline support at default which may be very useful for smart phones since smart phones frequently go offline and come back a later time.

Other use case is running pre-define queries on top of occasionally changing data.
Compared to ad-hoc query capabilities of MongoDB, views of CouchDB require a bit more design beforehand.
However, if data structure is hardly changing, data is accumulating and versions of data are also important, then CouchDB is the perfect solution.

CouchDB is in production at Credit Suisse, dotCloud and Engine Yard but there are also some failure stories such as Ubuntu One.

\begin{table}[!ht]
  \centering
  \caption{CouchDB Document-Oriented Database}
  \renewcommand{\arraystretch}{1.5}
  \begin{tabular}{| >{\centering\bfseries}m{1in} | >{\centering\arraybackslash}m{4.5in} |}
	\hline
    \textbf{Criteria} & \textbf{Feature} \\
	\hline
    Availability &
    CouchDB, licensed under Apache License 2.0, is one of the high priority projects in Apache Foundation and more mature compared to other NoSQL prouducts because it has longer history and is developed by more inter-company developers.
    \\ \hline
    Scalability &
    CouchDB shines at scalability with incremental multi-master replication and also implements MVCC to prevent writes to block reads. Sharding may be done in application level but there are couple of products to extend CouchDB such as auto-sharding by CouchBase.
    \\ \hline
    Ease of Use Development &
    It is developed for web and provides HTTP API so can be easily used with any language.
    Data is stored in documents as MongoDB which is more natural and these documents are very flexible due to lack of schema.
    \\ \hline
    Transactions Consistency &
    ACID-like eventual consistency is guaranteed.
    This is far from the transactions of relational world but closer than MongoDB to what is needed by Indico.
    \\ \hline
    Community Momentum &
    Compared to MongoDB, CouchDB has a smaller community and it is preferred by companies in more niche categories such as PaaS providers while MongoDB is in production at eBay, Foursquare, New York Times and SourceForge.
    \\ \hline
    Cost \\ Exit Strategy &
    Coming into CouchDB from object-oriented world is easier than relational since documents are similar to objects.
    Consistency guarantees of CouchDB is more analogous to ZODB than MongoDB but this makes CouchDB replacement more difficult compared to MongoDB with other document databases.
    \\ \hline
    Score & \rpt[1]{\FiveStarOpen}\rpt[5]{\FiveStar} \\
    \hline
  \end{tabular}
  \label{couchdb}
\end{table}

\section{Key-Value Stores}

Key-value stores are distributed hash table implementation for larger data.
Each object has a unique key to be accessed with and values can be anything, seen as pure byte streams, so key-value stores don't enforce any schema.
Moreover, object databases can be as an example of key-value stores since value is just an object of a specific programming language.

Key-value stores don't try to interpret bytes of values.
As much as they give meaning to the value, they are closer to document-oriented databases.
Nature of design makes them highly scalable but only scalability isn't enough because ease-of-use, usage difficulty and quering capabilities are also important.
The more ad-hoc query capabilities key-value store has, the closer it is to document-oriented databases and the more usable it is in general so different products try to give a compromise in-between.

Due to simpler architecture, this is the category that has the most number of products.
Products are very different in terms of consistency characteristics, key structure, back-end and sorted-ness of keys. Main competitors are BerkeleyDB, Hazelcast, LevelDB, redis, Riak, Voldemort, memcached, Tarantool. Even document-oriented databases, MongoDB and CouchDB, can be seen as very capable key-value stores.

\subsection{Redis}

Redis is a very simple key-value storage solution. It can be compared to memcached on steroids - a key-value store with extra [data structures] and additional features such as transactions and PubSub. Contrarily to e.g. memcached, it also allows developers to user server-side Lua scripts, akin to stored procedures. Redis will by default keep its data only in memory, but data can also be persisted on disk if desired. Redis Cluster is recently released to bring important features such as replication and strong consistency.

\begin{table}[!ht]
  \centering
  \caption{Redis Key-Value Store}
  \renewcommand{\arraystretch}{1.5}
  \begin{tabular}{| >{\centering\bfseries}m{1in} | >{\centering\arraybackslash}m{4.5in} |}
	\hline
    \textbf{Criteria} & \textbf{Feature} \\
	\hline
    Availability &
    According to rankings, Redis is the most popular key-value store in use and BSD license is suitable for Indico.
    However, Redis is still developed by only one developer, in the core.
    \\ \hline
    Scalability &
    Redis is developed as a single threaded in-memory database.
    Later, optional durability is added by flushing memory in intervals aka. snap-shotting or asynchronous append-only operation, \textit{journaling}.
    In December 2013, redis cluster, distributed version of Redis, is released but it is unstable and there is no success stories, at least for now.
    \\ \hline
    Ease of Use Development &
    Python is a well supported language.
    However, there is no rich data structure that can easily map objects of ZODB. Thus, converting complex hierarchy of objects to Redis primitives will be a pain and time taking.
    \\ \hline
    Transactions Consistency &
    Redis supports transactions but since it's single threaded, transactions are executed in a blocking fashion which deteriorates performance.
    It's also important to note that there is no roll-back for failed transactions.
    \\ \hline
    Community Momentum &
    Community size is medium since Redis can't be the only database system, it's more suited to work as a helper for another main storage engine but there are also a few successful only Redis deployments which have very very high write loads.
    Redis is in production at craiglist, flickr, stackoverflow.
    \\ \hline
    Cost \\ Exit Strategy &
    Using Redis is a bit difficult due to lack of complex structures but when time has come to exit, it is also difficult because Redis provides some powerful data structures compared to other key-value stores and if these structures are used, they may not be replaced easily.
    Moreover, main memory is the limiting factor for the size of the data because data managed by Redis can't be bigger than available memory.
    \\ \hline
    Score & \rpt[2]{\FiveStarOpen}\rpt[4]{\FiveStar} \\
    \hline
  \end{tabular}
  \label{redis}
\end{table}

\subsection{Riak}

Riak is an open-source fault-tolerant key-value store inspired by Amazon Dynamo. It's very configurable in terms of trade-off proved by CAP Theorem.

Riak replicates data in master-less fashion into $n_val$ nodes which is \textit{three} at default. Where data will be written is achieved by consistent hashing. In case of node failure or network partition, keys can be written to neighbouring nodes (hinted hand-off) and when failing nodes are back, new nodes off-load their part and data, updated while they are unavailable, is rewritten to them.

Reads requires $R$ number of nodes to be read so cluster can tolerate $N - R$ node failures.

Consistent hashing enables distributing data evenly which makes query latency very predictable even in node failures. To make division more evenly, if there are low number of physical nodes, each Riak physical node creates \textit{vnodes} virtual nodes. Moreover, it also enables assigning different amount of data to nodes by changing \textit{vnodes} for the respective physical nodes if physical nodes have different specifications (more RAM, SSD, etc.). Riak is totally designed for distributed environment so generally, adding more nodes makes operations faster.

All nodes are equal and there is no master so any request can be served by any node. Consistency of data under concurrent writes is achieved by vector clocks.

Like CouchDB, Riak is written in Erlang and fully talks REST and supports MapReduce via JavaScript. In addition, Riak leverages Apache Solr to provide search capabilities and links to traverse objects via MapReduce to provide graph-like features. Moreover, even if Riak is a Key-value that requires access to objects via keys, it has more than that. Riak uses multiple backends; namely, memory, Bitcask, LevelDB and any combination of them together. If LevelDB or memory is in use, objects can queried by secondary indexes which are integer or string values to tag objects. Queries support exact match or range retrieval. Result can be paginated, streamed or even be given as input into MapReduce job.

Like HBase, Riak support inter-cluster replication but Riak has a master where HBase is master-less. Replication is done in two modes, real time and full-sync in 6 hours.

Riak is the choice of the majority of first 100 traffic websites such as GitHub (URL shortener and pages) and Google via acquisitions.

\begin{table}[!ht]
  \centering
  \caption{Riak Key-Value Store}
  \renewcommand{\arraystretch}{1.5}% Spread rows out...
  \begin{tabular}{| >{\centering\bfseries}m{1in} | >{\centering\arraybackslash}m{4.5in} |}
	\hline
    \textbf{Criteria} & \textbf{Feature} \\
	\hline
    Availability &
    It has Apache License which is suitable for Indico and it is one of most popular key-value stores with Redis.
    \\ \hline
    Scalability &
    Riak is inspired by Amazon Dynamo paper.
    It's developed for scalability and high availability so it is very easy to add/remove machines to cluster.
    Inter-cluster replication is even possible in master-slave fashion in two modes, asynchronous and synchronous.
    Intra-cluster, every piece of data is stored in multiple nodes and consistent hashing is used to divide the data between machines so Riak has very predictable latency.
    \\ \hline
    Ease of Use Development &
    Python is one of the officially supported languages but mapping complex object structures to very primitive data types is difficult.
    Moreover, lack of proper transactions like in relational databases puts burden on readers.
    \\ \hline
    Transactions Consistency &
    Riak implements vector clocks and reads use them to decide which write is the latest under concurrent updates so reads require extra work but writes always succeed.
    \\ \hline
    Community Momentum &
    Riak is in production at very popular websites due to its highly fault-tolerant nature which is vital for a business takes huge traffic, a result of being big. \\ \hline
    Cost \\ Exit Strategy & Entering into Riak as a main storage is difficult due to lack of complex data structures because complex structures of Indico should be converted into simple ones. There is no direct replacement of Riak in provided distributed fault tolerancy support so leaving Riak is also difficult. \\ \hline
    Score & \rpt[2]{\FiveStarOpen}\rpt[4]{\FiveStar} \\
    \hline
  \end{tabular}
  \label{riak}
\end{table}

\section{Graph Databases}

Graph Databases focus on graph structured datasets where adjacency of data is important and relations between entities make them closer to each other.
Graph databases try to optimize retrieval of these relations.

Graph databases are composed of two main parts; underlying storage and processing engine:

\begin{table}[!ht]
  \centering
  \caption{Architecture of Graph Databases}
  \renewcommand{\arraystretch}{1.5}
  \begin{tabular}{| >{\centering\bfseries}m{2in} | >{\centering}m{1in} | >{\centering\arraybackslash}m{1in} |}
	\hline
    \textbf{Graph DB} & \textbf{Storage} & \textbf{Engine} \\ \hline
	Twitter FlockDB & non-native & non-native \\ \hline
	neo4j & native & native \\ \hline
	OrientDB & native & native \\ \hline
	Titan & non-native & native \\ \hline
  \end{tabular}
  \label{orientdb}
\end{table}

Underlying storage may be a native graph storage which relations directly point to physical location of entities. This is very different than foreign keys in RDBMS because there is no need of computation to retrieve related entities since entity itself contains links to physical location of its neighbours which can be used in \textit{O(1)} time.

Computing engine implements standard graph algorithms on top of the storage engine. Performance and capabilities of this layer highly depends on storage. If storage is a native graph, graph can't be easily separated into parts which hoists scalability concerns. Even if database supports distributed architecture, whole data must be replicated to each node. However, having non-native storage enables processing giant graphs in the order of trillion edges via sharding capabilities of lower level storage, Cassandra or HBase, for instance. Unless graph partition quality is enough, performance characteristics can have noticeable differences since while native storage has whole data via \textit{O(1)} access, non-native storage may need to talk to many nodes over network.

Supporting a standard graph processing interface is important because it makes transition easier between databases according to changing needs of the domain. Therefore, even if there are graph databases with non-native graph processing in the market, they aren't listed (FlockDB is an exception due to being open source but it seems to be dead at the time of writing due to activity and Scala version, main development language) because they are commercial, old (first examples of graph databases) and have much smaller user community.

Graph databases seems to be de-facto go to database in the implementation of personal dashboard and recommendations, and also next generation of features of Indico such as complete socialization and gamification.

\subsection{OrientDB}

OrientDB is a hybrid between document-oriented and graph databases. By default, it works like a Document-oriented DB in which records are JSON-like documents. However, there is a \textit{graph} layer implemented on top of this that allows for elaborate relations to be established between documents. In this layer, both vertex and edges are documents that can be transparently manipulated.

There are 3 versions of OrientDB: 
\begin{enumerate}
  \item Standard (Document-oriented and Graph)
  \item Graph (TinkerPop Stack)
  \item Enterprise (coming with auto sharding/replication capabilities)
\end{enumerate}

Graph functionality is actually included in the standard edition - it's a common misunderstanding. The graph edition differs from the the standard one in the fact that it includes a TinkerPop stack. OrientDB is 100\% compliant with this graph processing stack. OrientDB can also be used as a key-value store, since document-orientation is a super-set of key-value stores. That is made possible by allowing records to be documents or flat strings. The data can be fully kept in memory if desired.

We have played a bit more with OrientDB compared to other databases since its features seemed to be a better fit for Indico.

\subsubsection*{Concepts}

There are some concepts that make OrientDB quite different from its relational companions:

\begin{itemize}
  \item \textbf{Record} - an instance of data as row in RDBMS or document in document-oriented databases.
  \item \textbf{Record ID} - auto-assigned unique number. It directly specifies the place of a record in disk, so it's not the equivalent to a logical ID (primary key) in RDBMS. That's why retrieving a record, when the ID is known, works in constant time, while it's O(log(n)) in RDBMS, using an index based on the id.
  \item \textbf{Cluster} - collection of links to records. A Record ID is composed of a Cluster ID and a sequence number within the cluster.
  \item \textbf{Class} - an abstraction of cluster used to group records. However, being a member of class doesn't put any constraints on records. Classes actually work in a similar way to their counterparts in Object-Oriented Programming and can bring new functionality via inheritance. By default, there is 1-to-1 mapping between classes and clusters and clusters can be seen as a table in an RDBMS that is used to group same type instances. However, in advanced usage, mapping may be n-to-m. Some examples:
  \begin{itemize}
	\item cluster \textit{person} to group all records from \textit{person} class (1-to-1)
	\item cluster \textit{cache} to group most accessed records (1-to-n)
	\item cluster \textit{car} to group all records belonging to the "car" class by type; suv, truck, etc. (n-to-1)
	\item cluster \textit{daily} to group all records by creation day (n-to-m)
  \end{itemize}
\end{itemize}

\subsubsection*{Security}

OrientDB has powerful security mechanism compared to other NoSQL stores; namely, rule (bitmask for CRUD), role (group of rules) and user (executor of rules). Rules can be defined server, database, cluster or record level. These features brings OrientDB closer to RDBMS, which are strict and powerful in  security.

\subsubsection*{Transactions}

Since we heavily rely on transactions, transaction support is important in a database. MongoDB allows atomic operations at the document level. However, we may need operations that span multiple documents if the data is to be split across collections. On the contrary OrientDB is fully ACID-compliant:
\begin{itemize}
  \item It allows multiple reads and writes on the same server
    \begin{itemize}
      \item client-scoped and no lock on the server
      \item implemented by a special property $@version$ tracked for each record. Mismatch means roll-back (MVCC).
    \end{itemize}
    \item Distributed transactions aren't supported. However, OrientDB got master-less replication via Hazelcast with latest release and as seen from code base, they have the base for distributed transactions. It's seen a must for horizontal scaling by the development team and planned for $2.0$ version. Thus, it's only a matter of time.
    \item Nested transactions aren't supported. It's highly probable that they will be implemented while distributed transactions.
\end{itemize}

\subsubsection*{Speed}

OrientDB seems to be really fast compared to other graph databases: 2 to 3 times faster than Neo4j in some benchmarks. In a fairly recent analysis (2012) by the Tokyo Institute of Technology and IBM, OrientDB was considered the fastest graph database from a group that included AllegroGraph, Fuseki and Neo4j (market leader).

Why is OriendDB fast?

\begin{itemize}
  \item Cache
  \begin{itemize}
    \item Level 1 - Thread Level (for each open connection) in client and Database Level in server
    \item Level 2 - JVM Level (shared between all connections) in client and Storage Level in server
    \item Advanced Storage cache depends on implementation
    \item OS cache for Memory Mapped files
  \end{itemize}
  \item Indexing
  \begin{itemize}
    \item Tree Index - MVRB-Tree (proprietary but open source algorithm that combines best parts of RB-Tree and B+ Tree). Tree Index is heavier and works in $O(logn)$ time but enables range queries.
    \item Hash Index - Lighter and works in \textit{constant time}.
    \item Composite Index (usable partially)
  \end{itemize}
  \item Hard Links
  \begin{itemize}
    \item Record ID is a pointer to a real physical place, not a logical concept so unlike RDBMS, it needs no calculation to access. As a result, Data loading is $O(1)$ time by record id.
  \end{itemize}
  \item Memory-mapped files
  \begin{itemize}
    \item Java New I/O (JSR 51 aka. NIO)
    \item No system calls, since there is no switch between kernel and user modes
	\item In-place update (\textit{seek} only required if a record is beyond the current page's boundaries)
	\item Lazy loading (efficient usage of memory)
	\item Adjusted page size via collected statistics about database
	\item The downside is that 32-bit, being only to address at most 4 Gb in their virtual address space, are limited in terms of file size.
  \end{itemize}
  \item Transactions
  \begin{itemize}
    \item No lock for reads (MVCC)
  \end{itemize}
  \item Tools to tweak
  \begin{itemize}
    \item explain (standard as PostgreSQL or MongoDB)
    \item profiler
  \end{itemize}
\end{itemize}

\subsubsection*{Scaling}

Scaling is done by multi-master scheme, each instance auto-magically synchronize each other. There are two modes:
\begin{itemize}
  \item synchronous: slower writes. The larger cluster size is, the slower writes because writes wait for each node to acknowledge success and quorum as in Cassandra or Riak is not configurable for now.
  \item asynchronous: faster but naturally weaker consistency
\end{itemize}

Each node has the whole database which may be a problem in some cases if database is so large (in a 64-bit system, it is very unlikely since virtual address space huge enough). Sharding is possible by grouping nodes and forming clusters according to shard keys but that possibly would bring some logic to client for whom to talk.

\subsubsection*{API}

Java and Scala are the main supported languages. However, Python is also fully supported by 3 drivers:

\begin{itemize}
  \item Native driver, a wrapper over C library (liborient)
  \item HTTP REST driver
  \item \textit{Rexster} driver (only if graph edition is in use)
\end{itemize}

There is a powerful console (native) with auto-complete that makes trying commands/concepts fairly easy. Moreover, there is another open-source project called OrientDB studio to accomplish administrative tasks easier like what \textit{PHPMyAdmin} does for MySQL/MariaDB.

% SHOULD KEEP?
The company that is supporting the development of database and providing database in the cloud as a service, is closed to focus more on the development side. Whether it is a good thing is a big mystery because it is more likely that they couldn't keep up with a sustainable business due to market share of OrientDB and very established competitors.

\subsubsection*{Extensibility}

There are two ways supported:
\begin{itemize}
  \item Server-side functions (different than triggers): totally generic functions that are also capable of executing sql queries on the database.
  \begin{itemize}
    \item Java
    \item JavaScript on Mozilla Rhino
  \end{itemize}
  \item Hooks (Java plug-ins to core)
\end{itemize}

\subsubsection*{Backup/Restore}

There is import/export mechanism to easily move the data/schema or upgrade. Automatic regular backup is native but it should be enabled.

There is also an emailing module which may be used for emergency, statistics, etc.

\subsubsection*{Competitors}

There are a lot of graph databases but there are a few in active development; namely, DEX, Neo4j and Titan. DEX is out of comparison since it is commercial. Neo4j is the leader. Titan is quite new but could catch up with others. Titan goes pretty good with other systems such as Cassandra, HBase, ElasticSearch, etc. which is very important in big data era, to be able to deploy a polyglot storage system. Moreover, it is getting popular faster than OrientDB by providing better scalability with underlying storage engine.

\begin{table}[!ht]
  \centering
  \caption{OrientDB Graph Database}
  \renewcommand{\arraystretch}{1.5}% Spread rows out...
  \begin{tabular}{| >{\centering\bfseries}m{1in} | >{\centering\arraybackslash}m{4.5in} |}
	\hline
    \textbf{Criteria} & \textbf{Feature} \\
	\hline
    Availability &
    OrientDB is licensed under Apache License 2.0.
    \\ \hline
    Scalability &
    With newest release in November 2013, OrientDB supports multi-master replication via Hazelcast but sharding is offloaded to application. Within graph databases with native graph storage, OrientDB seems to be the fastest but for a healthier evaluation, comparison with other databases from different disciplines is missing.
    \\ \hline
    Ease of Use Development &
    Python is supported by drivers provided from community (binary or HTTP). Being document and graph database enables flexible schema which is easier for refactoring and intuitiveness.
    \\ \hline
    Transactions Consistency &
    ACID with MVCC \\ \hline
    Community Momentum &
    OrientDB is getting popular fairly quickly because it isn't a native graph database.
    It's a hybrid of document and graph databases where each node or edge is also a document as in MongoDB or CouchDB.
    Thus, it is able to provide benefits of both worlds at the same time.
    However, there aren't enough successful big deployments which brings the question of if OrientDB is really production ready.
    \\ \hline
    Cost \\ Exit Strategy &
    If it is compared to Neo4j, picking up OrientDB is easier because parts that are problematic in Neo4j can be mapped via more intuitive document nodes and edges.
    \\ \hline
    Score & \rpt[1]{\FiveStarOpen}\rpt[5]{\FiveStar} \\
    \hline
  \end{tabular}
  \label{orientdb}
\end{table}

\subsection{Neo4j}

% TODO: EXTEND NEO4J
%Pros:
%ACID
%High availability
%Most mature
%Biggest community and market leader (Cisco, Deutsche Telekom, HP, Huawei)
%Best search via external Lucene index
%Cons:
%Speed
%Less flexible properties compared to document nodes in OrientDB (cumbersome working with complex data)
%Dual license

\begin{table}[!ht]
  \centering
  \caption{Neo4j Graph Database}
  \renewcommand{\arraystretch}{1.5}
  \begin{tabular}{| >{\centering\bfseries}m{1in} | >{\centering\arraybackslash}m{4.5in} |}
	\hline
    \textbf{Criteria} & \textbf{Feature} \\
	\hline
    Availability &
    Neo4j is licensed under GPL and AGPL (for enterprise) and it is by far the most stable graph database in the market.
    \\ \hline
    Scalability &
    Scalability is lacking because horizontal scaling of a graph is really difficult due to relationships in a tightly connected graph.
    \\ \hline
    Ease of Use Development &
    Some parts of Indico such as roles of users in events, links between conferences, sessions and contributions, etc. perfectly match links in a graph but not everything is mappable.
    \\ \hline
    Transactions Consistency &
    Neo4j has ACID transactions like relational databases. \\ \hline
    Community Momentum &
    Neo4j is the most popular graph product but in general, graph databases are niche products and mainly used by communication companies, where connections between entities are deadly important to business.
    \\ \hline
    Cost \\ Exit Strategy &
    Starting to use Neo4j isn't difficult due to 1-to-1 mapping from ZODB objects to vertexes and edges of Neo4j.
    Graph databases supports \textit{Gremlin}, a graph traversing and manipulation language like SQL in relational world.
    If Gremlin is used, Gremlin creates an abstraction layer on back-end storage like \textit{SQLAlchemy} does for relational databases so back-end can easily be changed with another Gremlin-compliant database such as OrientDB or Titan.
    \\ \hline
    Score & \rpt[2]{\FiveStarOpen}\rpt[4]{\FiveStar} \\
    \hline
  \end{tabular}
  \label{neo4j}
\end{table}

\subsection{Titan}

% TODO: EXTEND TITAN
%Pros:

%ACID, vertex consistent or eventually consistent (according to storage)
%Multiple backends (Cassandra, HBase, BerkeleyDB)
%High availability and scability (standing on shoulders of giants)
%Native TinkerPop stack
%Search (Elastic or Lucene)
%Cons:

%Complexity of another storage
%Gaining popularity but small user base (Cisco, Pearson)

\begin{table}[!ht]
  \centering
  \caption{Titan Graph Database}
  \renewcommand{\arraystretch}{1.5}
  \begin{tabular}{| >{\centering\bfseries}m{1in} | >{\centering\arraybackslash}m{4.5in} |}
	\hline
    \textbf{Criteria} & \textbf{Feature} \\
	\hline
    Availability &
    Apache License 2.0 \\ \hline
    Scalability &
    Titan has a notably different architecture compared to other graph databases via pluggable storage back-ends where scalable back-ends such as HBase or Cassandra brings scalable nature into Titan.
    \\ \hline
    Ease of Use Development &
    Python is officially supported in binary protocol but while pluggable storage engine, which itself isn't easy to use, is bringing scalability, it also makes the system much more complex compared to OrientDB or Neo4j due to many moving parts.
    \\ \hline
    Transactions Consistency &
    ACID and eventual consistency are possible but it mainly depends on the back-end characteristics.
    \\ \hline
    Community Momentum &
    Graph databases are on the rise and data is getting bigger and bigger. Other graph databases can't divide their data but Titan can divide by following a different path. Thus, Titan can scale to the numbers OrientDB and Neo4j can't. If GitHub popularity of graph databases is ranked (all three are hosted on GitHub), then Titan seems to be the most popular one.
    \\ \hline
    Cost \\ Exit Strategy &
	Start phase to use Titan may be probably difficult since required storage engine. Exit is easier comparably since Titan is also 100\% compliant with TinkerPop stack. If application is using TinkerPop stack, when graph databases is changed, there is nothing to be changed from the application point of view.
    \\ \hline
    Score & \rpt[2]{\FiveStarOpen}\rpt[4]{\FiveStar} \\
    \hline
  \end{tabular}
  \label{titan}
\end{table}

\subsection{Broadness and Validity of Survey}

We try to give basic architecture details of each database but there are normally much more details to be considered. However, database ecosystem is quickly changing. We have seen multiple versions same database even while this writing. That's why we tried to keep more tiny details online in a living document. This enables us to easily update data and since it gives comparison more exposure, small mistakes are found and fixed earlier.

% ---------------------------------------------------------------------------
% ----------------------- end of thesis sub-document ------------------------
% ---------------------------------------------------------------------------


% --------------------------------------------------------------
%:                  BACK MATTER: appendices, refs,..
% --------------------------------------------------------------

% the back matter: appendix and references close the thesis


%: ----------------------- bibliography ------------------------

% The section below defines how references are listed and formatted
% The default below is 2 columns, small font, complete author names.
% Entries are also linked back to the page number in the text and to external URL if provided in the BibTex file.

% PhDbiblio-url2 = names small caps, title bold & hyperlinked, link to page 
%\begin{multicols}{2} % \begin{multicols}{ # columns}[ header text][ space]
\begin{tiny} % tiny(5) < scriptsize(7) < footnotesize(8) < small (9)

\bibliographystyle{Latex/Classes/PhDbiblio-url2} % Title is link if provided
\renewcommand{\bibname}{References} % changes the header; default: Bibliography
   
\bibliography{backmatter/references} % adjust this to fit your BibTex file

\end{tiny}
%\end{multicols}  

% --------------------------------------------------------------
% Various bibliography styles exit. Replace above style as desired.

% in-text refs: (1) (1; 2)
% ref list: alphabetical; author(s) in small caps; initials last name; page(s)
%\bibliographystyle{Latex/Classes/PhDbiblio-case} % title forced lower case
%\bibliographystyle{Latex/Classes/PhDbiblio-bold} % title as in bibtex but bold
%\bibliographystyle{Latex/Classes/PhDbiblio-url} % bold + www link if provided

%\bibliographystyle{Latex/Classes/jmb} % calls style file jmb.bst
% in-text refs: author (year) without brackets
% ref list: alphabetical; author(s) in normal font; last name, initials; page(s)

%\bibliographystyle{plainnat} % calls style file plainnat.bst
% in-text refs: author (year) without brackets
% (this works with package natbib)


% --------------------------------------------------------------

% according to Dresden med fac summary has to be at the end
%
% Abstract -------------------------------------------------------------


%\begin{abstractslong}    %uncommenting this line, gives a different abstract heading
\begin{abstracts}        %this creates the heading for the abstract page

\par Indico is a web application to schedule and organise events, from simple lectures to complex meetings, workshops and conferences with sessions and contributions. The tool also includes advanced user delegation mechanism, paper reviewing, archiving of event materials. More tailored features such as room booking and collaboration are provided via plug-ins.
\\
\par Indico is heavily used at CERN and in more than 10 years, very different features are added according to the needs of increasing number of users. However, data store(ZODB) have never changed. ZODB has been very important for Indico to have been Indico but it's a bottleneck now in terms of scalability and elapsed time in feature development.
\\
\par Indico demands more scalable backend to serve increasing number of events with fancy features. The aim of the project is to replace ZODB with a highly scalable data layer which will enable global Indico, bringing $\sim$120 different Indico servers around the world together. First part of project involves the survey of different technologies according to the needs of Indico and second part is composed of preparation of transition environment with a prototype.
\\
\\
\textbf{Keywords}: SQL, NoSQL, Python, Web Development

\end{abstracts}

%\end{abstractslong}


% ---------------------------------------------------------------------- 


%: Declaration of originality
%\begin{declaration}

I herewith declare that I have produced this paper without the prohibited assistance of third parties and without making use of aids other than those specified; notions taken over directly or indirectly from other sources have been identified as such. This paper has not previously been presented in identical or similar form to any other examination board. \\

The thesis work was conducted from 17 September 2013 to 14 Mar 2014 under the supervision of Pedro Ferreira at CERN and Karl Aberer at EPFL.

\vspace{8mm}

GENEVA, Switzerland

\end{declaration}




\end{document}
